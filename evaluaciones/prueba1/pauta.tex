\documentclass{article}
\usepackage{fullpage}
\usepackage{mathpazo}
\usepackage{todonotes}
\usepackage{minted}

\newcommand{\note}[1]{\todo[inline,color=gray!20!white]{#1}}
\newcommand{\code}[1]{\texttt{#1}}
\newcommand{\alternatives}[5]{
    \begin{enumerate}
        \item #1
        \item #2
        \item #3
        \item #4
        \item #5
    \end{enumerate}
}

\title{Interrogación I}
\author{Nicolás Barnafi, Humberto Reyes}
\date{13/09/2024}

\begin{document}
\maketitle
Nombre completo: 

\section*{Parte 1}
Responda las siguientes preguntas de alternativas. Cada pregunta correcta vale X puntos. 

    \begin{enumerate}
        \item Cuál es el tipo del resultado de esta operación: \texttt{ 3 / 2}

        \alternatives{\code{string}}{\code{\bf float}}{\code{int}}{\code{bool}}{\code{list}}

        \item Cuál es el tipo del resultado de esta operación: \texttt{3==2}
    
        \alternatives{\code{string}}{\code{float}}{\code{int}}{\code{\bf bool}}{\code{list}}

        \item Cuál es el tipo del resultado de esta operación: \texttt{ 3 \% 2}

        \alternatives{\code{string}}{\code{float}}{\code{\bf int}}{\code{bool}}{\code{list}}

        \item Cuál es el tipo del resultado de esta operación: \texttt{ls[1]}

        donde \code{ls = [12, True, 'hola']}.

        \alternatives{\code{string}}{\code{float}}{\code{int}}{\code{\bf bool}}{\code{list}}

        \item Python es un lenguaje:

        \alternatives{Primitivo}{Compilado}{\bf Interpretado}{Poco usado}{No orientado a objetos}

        \item Escoja es la forma correcta de ejectutar el script \code{test.py} en Bash:

        \alternatives{\code{\$ python test}}{\code{\$ python exec test.py}}{\code{\$ ./test.py}}{\code{\bf \$ python test.py}}{\code{\$ python test}}

        \item Elija el comando que sirve para ver los contenidos de un directorio

        \alternatives{\code{cat}}{\code{\bf ls}}{\code{cat}}{\code{mkdir}}{\code{nano}}

        \item Elija el comando que sirve para ver los contenidos de un archivo de texto

        \alternatives{\code{\bf cat}}{\code{ls}}{\code{cat}}{\code{mkdir}}{\code{nano}}

        \item Elija el comando que sirve para editar los contenidos de un archivo de texto

        \alternatives{\code{cat}}{\code{ls}}{\code{cat}}{\code{mkdir}}{\code{\bf nano}}

        \item Elija el comando que sirve para crear una carpeta nueva

        \alternatives{\code{cat}}{\code{ls}}{\code{cat}}{\code{\bf mkdir}}{\code{nano}}
    \end{enumerate}

\newpage
\section*{Parte 2}
A partir del código a continuación:
    \begin{minted}{python}
        def calcularIndicadores(xs, out='P'):
            def promedio(_xs):
                s = sum(_xs)
                N = len(_xs)
                return s / N
            def var(_xs):
                prom = promedio(_xs)
                suma = 0
                for x in _xs: 
                    suma += (x - prom)**2
                return 1/len(_xs) * suma
            if out == 'P':
                return promedio(xs)
            elif out == 'V':
                return var(xs)
            elif out == 'PV':
                return promedio(xs), var(xs)
            else: 
                print("No implemetado\n")
            
        def generarDatos(datos):
            out = {"datos": datos}
            p, v = calcularIndicadores(datos, out="PV")
            out["promedio"] = p
            out["varianza"] = v
            return out
    \end{minted}

Escriba la salida obtenida al ejecutar los siguientes comandos, considerando la variable \code{d = [1,3,5]}. Si genera un error, no es importante que muestre el error exacto si no que explique por qué se genera el error. 
    \begin{enumerate}
        \item print(calcularIndicadores(d))
        \hfill\textbf{3.0 }
        \item print(calcularIndicadores(d, out='P'))
        \hfill\textbf{3.0 }
        \item print(calcularIndicadores(d, out='V'))
        \hfill\textbf{2.6666666666666665 }
        \item print(calcularIndicadores(d, out='VP'))
        \hfill\textbf{No implemetado\textbackslash nNone }
        \item print(calcularIndicadores(d, out='PV'))
        \hfill\textbf{ (3.0, 2.6666666666666665) }
        \item print(generarDatos(d)["datos"])
        \hfill\textbf{[1, 3, 5] }
        \item print(generarDatos(d)["promedio"])
        \hfill\textbf{ 3.0 }
        \item print(generarDatos(d)[promedio])
        \hfill\textbf{No existe la variable \emph{promedio}}
        \item print(generarDatos(d)[1])
        \hfill\textbf{'1' no es una llave del diccionario de salida 'out'}
        \item print(generarDatos(d)["varianza"])
        \hfill\textbf{ 2.6666666666666665 }
    \end{enumerate}

\newpage
\section*{Parte 3}
Este código fue visto en clases como ejemplo del ciclo \code{while}, y muestra un algoritmo que representa cómo se puede ahorrar dinero a través de trabajo remunerado diario.

    \begin{small}
    \begin{minted}{python}
        plata = 0
        dias = 0 # contador
        while True:
            if dia == "sabado" or dia == "domingo":
                dia = avanzarDia(dia)
                dias += 1
                continue # Volver al inicio del while
            plata += trabajar()
            if plata >= objetivo:
                break # Salir del while
            dia = avanzarDia(dia)
            dias += 1
        print("Meta alcanzada en {} días".format(dias))
    \end{minted}
    \end{small}

Este código actualmente no está completo, por lo que no se podría ejecutar. Entregue el detalle de todo lo que falta para que este código se pueda ejecutar y que cumpla con su objetivo. Proponga además una variación del código donde la ganancia generada en cada día sea diferente. 

\begin{itemize}
    \item  \textbf{(2 pts) definir el dia de partida. }

    \begin{minted}{python}
        dia = 'lunes'  (puede ser cualquiera)
    \end{minted}

    \item \textbf{(4 pts) definir la función 'avanzarDia' }

    \begin{minted}{python}
        def avanzarDia(dia):
            if dia == 'lunes': return 'martes'
            if dia == 'martes': return 'miercoles'
            if dia == 'miercoles': return 'jueves'
            if dia == 'jueves': return 'viernes'
            if dia == 'viernes': return 'sabado'
            if dia == 'sabado': return 'domingo'
            if dia == 'domingo': return 'lunes'
    \end{minted}
    
    \item \textbf{(4 pts) definir función 'trabajar' que entrega algún monto de dinero. }

    \begin{minted}{python}
        def trabajar(): return 100000
    \end{minted}

    \item \textbf{(2 pts) definir variable 'objetivo' }

    \begin{minted}{python}
        objetivo = 1e6 
    \end{minted}

    \item \textbf{(8 pts) Extender el código: }

    \begin{minted}{python}
    def trabajar(dia): 
        if dia == 'lunes': return 1000
        if dia == 'martes': return 900
        if dia == 'miercoles': return 800
        if dia == 'jueves': return 700
        if dia == 'viernes': return 600
    \end{minted}
    
    \textbf{Cualquier extensión que efectivamente dependa del día y que entregue alguna cantidad según eso, estará bien. El de arriba es un ejemplo simple y válido.}
\end{itemize}
     

\end{document}
