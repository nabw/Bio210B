\documentclass{article}
\usepackage{fullpage}
\usepackage{mathpazo}
\usepackage{todonotes}
\usepackage{minted}
\usepackage{mathpazo}
\usepackage{todonotes}
\usepackage{forest}
\usepackage{amsmath}

\newcommand{\note}[1]{\todo[inline,color=gray!20!white]{#1}}
\newcommand{\code}[1]{\texttt{#1}}
\newcommand{\ind}{{\,\,\,\,}}


\title{Interrogación II, BIO210B}
\author{Nicolás Barnafi, Humberto Reyes}
\date{30/10/2024}

\begin{document}
Nombre completo: 
\hrule
{\let\newpage\relax\maketitle}


%%%%%%%%%%%%%%%%%%%%%%%%%%%%%%%%%%%%%%%%%%%%%%%%%%%%%%%%%%%%%%%%%%%%
%%%%%%%%%%%%%%%%%%%%%%%%%%%%%%%%%%%%%%%%%%%%%%%%%%%%%%%%%%%%%%%%%%%%
\section*{Pregunta 1}
%%%%%%%%%%%%%%%%%%%%%%%%%%%%%%%%%%%%%%%%%%%%%%%%%%%%%%%%%%%%%%%%%%%%
%%%%%%%%%%%%%%%%%%%%%%%%%%%%%%%%%%%%%%%%%%%%%%%%%%%%%%%%%%%%%%%%%%%%
    Escriba los contenidos del script 'test.py' de manera que cumpla con los siguientes requisitos: 
    \begin{enumerate}
        \item Que funcione como se espera al ser ejecutada desde la carpeta 'home'. En otras palabras, no considere la dificultad de ejecutar test.py desde otra carpeta.
        \item Que utilice todas las funcionalidades contenidas en los módulos en la carpeta 'modulos'.
        \item Que combine los archivos 'd1.xls' y 'd2.xls' en un único archivo 'd.xls' que sea guardado en la carpeta 'resultados/datos'. Para esto, debe cargar los datos usando el módulo 'leer' y luego procesarlos con el módulo 'm1'.

        \item Que lo primero que muestre el script sea el saludo entregado por el módulo 'hola'.
        \item Que procese los datos obtenidos en el paso 3 y que los guarde en un archivo 'resultado.txt' en la carpeta 'resultados'.
        \item Que concluya con el mensaje de finalización implementado en 'hola'.
    \end{enumerate}
Para esto, deben leer los datos dando los archivos conocidos. Una posible implementación podría ser

\begin{minted}{python}
from modulos.io import leer
import hola
from modulos import m1
hola.saludar()
d1 = "datos/d1.xls"
d2 = "datos/d2.xls"
out = "resultados/datos/d.xls"
datos = leer.leerDatos(out, [d1,d2] )
m1.procesar_datos(datos, "resultados/resultado.txt")
hola.finalizar()
\end{minted}

Puntos: 
\begin{itemize}
    \item 0.5 puntos por importar bien 'hola'.
    \item 0.5 puntos por importar bien 'leer'.
    \item 0.5 puntos por importar bien 'm1'.
    \item 1.0 punto por llamar correctamente 'leerDatos'.
    \item 1.0 punto por llamar correctamente 'procesar\_datos'.
    \item 1.0 punto por llamar correctamente 'saludar' y 'finalizar'.
\end{itemize}
%%%%%%%%%%%%%%%%%%%%%%%%%%%%%%%%%%%%%%%%%%%%%%%%%%%%%%%%%%%%%%%%%%%%
%%%%%%%%%%%%%%%%%%%%%%%%%%%%%%%%%%%%%%%%%%%%%%%%%%%%%%%%%%%%%%%%%%%%
\section*{Pregunta 2}
%%%%%%%%%%%%%%%%%%%%%%%%%%%%%%%%%%%%%%%%%%%%%%%%%%%%%%%%%%%%%%%%%%%%
%%%%%%%%%%%%%%%%%%%%%%%%%%%%%%%%%%%%%%%%%%%%%%%%%%%%%%%%%%%%%%%%%%%%
Considere un archivo 'csv' con el siguiente formato dado por dos columnas de números:
    \begin{verbatim}
'x','y'
12,4.3
2.1,-4
5,7.
2,3
    \end{verbatim}
El archivo puede tener un número arbitrario de filas. Escriba un script en Python que realice lo siguiente:
    \begin{itemize}
        \item Tiene que cargar los datos del archivo 'csv' en dos listas separadas, una para los datos en la columna 'x' y otra para 'y'.
        \item Debe graficar los datos considerando que 'y' es una función de 'x', i.e. $y=y(x)$.
        \item Debe finalizar imprimiendo la media y varianza de ambas variables ('x' e 'y'), y al final mostrar la correlación entre ambas. Dejamos las fórmulas para sus cálculos como referencia: 
            $$ \texttt{media}(X) = \frac 1 N \sum_i x_i \qquad \texttt{var}(X) = \sigma_X^2 = \frac{1}{N-1}\sum_i(x_i-\texttt{media(X)})^2$$
            $$\texttt{corr}(X,Y) = \frac{\sum_i\left(x_i - \texttt{media}(X))(y_i - \texttt{media}(Y))\right)}{\sigma_X\sigma_Y} $$
    \end{itemize}
Puede usar cualquiera de las librerías vistas en el curso para el desarrollo de este ejercicio.

\note{
\code{\# Se puede leer con otras librerías}

\code{f = open("archivo.csv", 'r') \# asumir cualquier nombre}

\code{lines = f.readlines()[1:]}

\code{x = [ float(split(l, ',')[0]) for l in lines]}

\code{y = [ float(split(l, ',')[1]) for l in lines]}

\code{import matplotlib.pyplot as plt}

\code{plt.plot(x,y)}

\code{plt.show()}

\code{\# Código para cálculos finales}

\code{sumax = sumay = 0}

\code{N = len(x)}

\code{for i in range(N): }

\code{    sumax += x[i]}

\code{    sumay += y[i]}

\code{mediax = sumax / N}

\code{mediay = sumay / N}

\code{sumax = sumay = 0}

\code{for i in range(N):}

\code{    sumax += (x[i] - mediax)**2}

\code{    sumay += (y[i] - mediay)**2}

\code{varx = sumax / (N-1)}

\code{vary = sumay / (N-1)}

\code{suma = 0}

\code{for i in range(N):}

\code{    suma += (x[i] - mediax) * (y[i] - mediay)}

\code{corr = corr / varx / vary}

\code{print("Medias en x e y: ", mediax, mediay)}

\code{print("Varianzas en x e y: ", varx, vary)}

\code{print("Correlación: ", corr)}
}

Distribución de los 4.5 puntos:
\begin{itemize}
    \item 0.5 por importar correctamente el archivo.
    \item 0.5 por importar la librería correspondiente.
    \item 1.0 por generar correctamente las listas de x e y.
    \item 1.0 por plotear correctamente (plot y show).
    \item 0.5 por calcular bien las medias
    \item 1.0 por calcular bien las varianzas
    \item 1.0 por calcular bien la correlación
\end{itemize}
%%%%%%%%%%%%%%%%%%%%%%%%%%%%%%%%%%%%%%%%%%%%%%%%%%%%%%%%%%%%%%%%%%%%
%%%%%%%%%%%%%%%%%%%%%%%%%%%%%%%%%%%%%%%%%%%%%%%%%%%%%%%%%%%%%%%%%%%%
\section*{Pregunta 3}
%%%%%%%%%%%%%%%%%%%%%%%%%%%%%%%%%%%%%%%%%%%%%%%%%%%%%%%%%%%%%%%%%%%%
%%%%%%%%%%%%%%%%%%%%%%%%%%%%%%%%%%%%%%%%%%%%%%%%%%%%%%%%%%%%%%%%%%%%
Considere una clase 'Vec3D' que genera el siguiente mensaje de ayuda:
    \begin{minted}{bash}
Help on class Vec3D in module __main__:

class Vec3D(builtins.object)
 |  Vec3D(x, y, z)
 |  
 |  Esta clase implementa un vector en tres dimensiones
 |  con varias funcionalidades adicionales, representado 
 |  como 
 |      | x_1 |
 |  x = | x_2 |
 |      | x_3 |
 |  
 |  Methods defined here:
 |  
 |  __add__(self, v2)
 |      Esta función crea un nuevo vector que tiene como 
 |      componentes la suma de las coordenadas de los
 |      dos vectores que se suman.
 |  
 |  __getitem__(self, i)
 |      Esta función permite acceder a las coordenadas 
 |      del vector como si fuera una lista, i.e.
 |      > v[1] # Segunda componente
 |  
 |  __init__(self, x, y, z)
 |      Constructor de clase. Recibe tres números que 
 |      serán las componentes x_1,x_2,x_3 del vector.
 |  
 |  norm(self)
 |      Esta función calcula la norma del vector
 |      y la entrega.

    \end{minted}
Escriba una clase 'Vec3D' que genere el mensaje de ayuda mostrado, y además implemente las funciones \texttt{\_\_init\_\_}, \texttt{\_\_add\_\_}, y \texttt{norm} para que sean consistentes con el docstring entregado. Dejamos como referencia que la norma de un vector $\vec X = (x,y,z)$ se define de la siguiente manera:
    $$ \| \vec X \| = \sqrt{x^2 + y^2 + z^2} . $$

\note{
\code{class Vec3D:}

\code{\ind"""}

\code{\ind Esta clase implementa un vector en tres dimensiones}

\code{\ind con varias funcionalidades adicionales, representado}

\code{\ind como}

\code{\ind x = (x1,x2,x3)  } 

\code{\ind """}

\code{\ind def \_\_add\_\_(self, v2):}

\code{\ind\ind""" }

\code{\ind\ind Esta función crea un nuevo vector que tiene como}

\code{\ind\ind componentes la suma de las coordenadas de los}

\code{\ind\ind dos vectores que se suman}

\code{\ind\ind""" }

\code{\ind def \_\_getitem\_\_(self, i):}

\code{\ind\ind""" }

\code{\ind\ind Esta función permite acceder a las coordenadas}

\code{\ind\ind del vector como si fuera una lista, i.e.}

\code{\ind\ind > v[1] \# Segunda componente}

\code{\ind\ind""" }

\code{\ind def \_\_init\_\_(self, x,y,z):}

\code{\ind\ind""" }

\code{\ind\ind Constructor de clase. Recibe tres números que}

\code{\ind\ind serán las componentes x\_1,x\_2,x\_3 del vector.}

\code{\ind\ind""" }

\code{\ind def norm(self):}

\code{\ind\ind""" }

\code{\ind\ind Esta función calcula la norma del vector}

\code{\ind\ind y la entrega.}

\code{\ind\ind""" }
}
La implementación de las funciones puede ser la siguiente: 

\begin{minted}{python}
def __init__(self, x,y,z):
    # Omitimos docstring acá
    self.x = x
    self.y = y
    self.z = z

def __add__(self, v2):
    new = Vec3D(self.x,self.y,self.z)
    new.x += v2.x
    new.y += v2.y
    new.z += v2.z
    return new

def norm(self):
    return (self.x**2 + self.y**2 + self.z**2)**(1/2)
\end{minted}

Distribución de los 3.5 puntos: 
\begin{itemize}
    \item 0.5 por cada comentario puesto en el lugar correct (suma 2.0)
    \item 0.5 por la implementación de cada función (suma 1.5)
\end{itemize}
%%%%%%%%%%%%%%%%%%%%%%%%%%%%%%%%%%%%%%%%%%%%%%%%%%%%%%%%%%%%%%%%%%%%
%%%%%%%%%%%%%%%%%%%%%%%%%%%%%%%%%%%%%%%%%%%%%%%%%%%%%%%%%%%%%%%%%%%%
\section*{Pregunta 4}
%%%%%%%%%%%%%%%%%%%%%%%%%%%%%%%%%%%%%%%%%%%%%%%%%%%%%%%%%%%%%%%%%%%%
%%%%%%%%%%%%%%%%%%%%%%%%%%%%%%%%%%%%%%%%%%%%%%%%%%%%%%%%%%%%%%%%%%%%
Asuma que los datos están en un archivo 'atmosfera.csv'. Escriba una rutina en R que ajuste un modelo lineal de la temperatura en función de los otros campos, asumiendo los siguientes datos empíricos (inventados para esta prueba): 
    \begin{itemize}
        \item La temperatura es proporcional al cuadrado de la concentración de nitrógeno.
        \item La temperatura es inversamente proporcional a la concentración de oxígeno.
        \item La temperatura es proporcional al logaritmo de la concentración de argón.
        \item La temperatura es proporcional a la concentración de $CO_2$ a la cuarta potencia.
    \end{itemize}

Una posible implementación de sería:

\begin{minted}{R}
data = read_csv("atmosfera.csv")

model = lm(T ~ I(N^2) + I(O^(-1)) + log(A) + I(CO2^4), data = data)
\end{minted}

Distribución de los 1.5 punto de la pregunta:
\begin{itemize}
    \item 0.5 por cargar bien los datos
    \item 1.0 por escribir bien el modelo
\end{itemize}
%%%%%%%%%%%%%%%%%%%%%%%%%%%%%%%%%%%%%%%%%%%%%%%%%%%%%%%%%%%%%%%%%%%%
%%%%%%%%%%%%%%%%%%%%%%%%%%%%%%%%%%%%%%%%%%%%%%%%%%%%%%%%%%%%%%%%%%%%
\section*{Pregunta 5}
%%%%%%%%%%%%%%%%%%%%%%%%%%%%%%%%%%%%%%%%%%%%%%%%%%%%%%%%%%%%%%%%%%%%
%%%%%%%%%%%%%%%%%%%%%%%%%%%%%%%%%%%%%%%%%%%%%%%%%%%%%%%%%%%%%%%%%%%%
Para esta pregunta, (1) explique lo que hace la función 'reproducir' y (2) escriba el código que sería necesario agregar a 'run.py' para generar cuatro especies distintas (que llamaremos de primera generación). Luego, que se reproduzcan las especies de primera generación de a pares para generar una segunda generación. Finalmente, obtenga un descendiente final entre las especies de segunda generación.

\note{La función 'reproducir' toma dos especies, y a partir de ellas genera una nueva cuya cadena de bases nitrogenadas que sirve para entregar una especie nueva con la mitad de cada cadena de bases nitrogenadas correspondientes a los progenitores.}

Una posible implementación del código sería la siguiente: 
\begin{minted}{python}
# Primera generación
e1 = Especie('ATA')
e2 = Especie('ATGCGTAGCT')
e3 = Especie('CG')
e4 = Especie('TTTTTTTTTTT')
# Segunda generación
e5 = reproducir(e1,e2)
e6 = reproducir(e3,e4)
# Tercera
e7 = reproducir(e5,e6)
\end{minted}

Distribución de 4.0 puntos:
    \begin{itemize}
        \item 1.0 por explicar lo que hace la función reproducir.
        \item 1.0 por generar correctamente 4 especies
        \item 1.0 por generar la segunda generación con progenitores correctos
        \item 1.0 por generar la tercera generación con progenitores correctos
    \end{itemize}
%%%%%%%%%%%%%%%%%%%%%%%%%%%%%%%%%%%%%%%%%%%%%%%%%%%%%%%%%%%%%%%%%%%%
%%%%%%%%%%%%%%%%%%%%%%%%%%%%%%%%%%%%%%%%%%%%%%%%%%%%%%%%%%%%%%%%%%%%
\section*{Pregunta 6}
%%%%%%%%%%%%%%%%%%%%%%%%%%%%%%%%%%%%%%%%%%%%%%%%%%%%%%%%%%%%%%%%%%%%
%%%%%%%%%%%%%%%%%%%%%%%%%%%%%%%%%%%%%%%%%%%%%%%%%%%%%%%%%%%%%%%%%%%%

Una posible implementación sería la siguiente: 
\begin{minted}{python}
a0 = 1.2
b0 = 0.779
ca,da,cb,db = 0.9,0.2,0.1,0.8

as = [a0]
bs = [b0]

for i in range(50):
    anew = ca * as[i] + da * bs[i]
    bnew = cb * as[i] + db * bs[i]
    as.append(anew)
    bs.append(bnew)

import matplotlib.pyplot as plt
plt.plot(as, label="a")
plt.plot(bs, label="b")
plt.show()
\end{minted}
Distribución de 3.5 puntos: 
\begin{itemize}
    \item 0.5 por definir los parámetros iniciales
    \item 0.5 por inicializar una estructura de datos para cada variable
    \item 1.0 por crear un ciclo iterativo donde se almacenen las variables a,b
    \item 0.5 por almacenar los valores de las variables en el tiempo  
    \item 1.0 por graficar
\end{itemize}
%%%%%%%%%%%%%%%%%%%%%%%%%%%%%%%%%%%%%%%%%%%%%%%%%%%%%%%%%%%%%%%%%%%%
%%%%%%%%%%%%%%%%%%%%%%%%%%%%%%%%%%%%%%%%%%%%%%%%%%%%%%%%%%%%%%%%%%%%
\section*{Pregunta 7}
%%%%%%%%%%%%%%%%%%%%%%%%%%%%%%%%%%%%%%%%%%%%%%%%%%%%%%%%%%%%%%%%%%%%
%%%%%%%%%%%%%%%%%%%%%%%%%%%%%%%%%%%%%%%%%%%%%%%%%%%%%%%%%%%%%%%%%%%%
Los pasos para arreglar el archivo son los siguientes: 
\begin{minted}{bash}
# clonar el repo
git clone git@github.com:empresa/software-empresa.git

# Reemplazar datos, dejamos eso como una función abstracta
reemplazarMayo3PorMayo4()

# Dejar nuevo cambio en "stage"
git add cumple.txt

# Hacer commit para actualizar repo local
git commit -m "corregir cumpleaños josé"

# Subir los cambios
git push
\end{minted}

Esta pregunta tiene 4.0 puntos, y es 1.0 punto por cada comando de git bien ejecutado.
%%%%%%%%%%%%%%%%%%%%%%%%%%%%%%%%%%%%%%%%%%%%%%%%%%%%%%%%%%%%%%%%%%%%
%%%%%%%%%%%%%%%%%%%%%%%%%%%%%%%%%%%%%%%%%%%%%%%%%%%%%%%%%%%%%%%%%%%%
\end{document}
