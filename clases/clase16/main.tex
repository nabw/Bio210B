\documentclass[14pt,aspectratio=169,xcolor=dvipsnames]{beamer}
\usetheme{SimplePlus}
\usepackage{booktabs}
\usepackage{minted}

\title[short title]{Clase 16: Clases y polimorfismo II}
\subtitle{}
\author[NA Barnafi] {Nicolás Alejandro Barnafi Wittwer}
\institute[UC|CMM] 
{
    Pontificia Universidad Católica de Chile \\
    Centro de Modelamiento Matemático
}

\titlegraphic{
    \vspace{-1.8cm}
    \begin{flushright}
      \includegraphics[height=2.5cm]{../images/logos/puc.png} 
    \end{flushright}
}

\date{09/10/2024}
%\setbeamercovered{transparent}

\begin{document}
%%%%%%%%%%%%%%%%%%%%%%%%%%%%%%%%%%%%%%%%%%%%%%%%%%%%%%%
\begin{frame}
    \maketitle
\end{frame}
%%%%%%%%%%%%%%%%%%%%%%%%%%%%%%%%%%%%%%%%%%%%%%%%%%%%%%%
\begin{frame}\frametitle{Motivación}
    \begin{itemize}
        \item Abstraer "Figura"
        \item Sistema ecológico (pizarra)
    \end{itemize}
    $$ \text{Carnívoro}\quad\to\quad\text{Herbívoro}\quad\to\quad\text{Plantas} $$
\end{frame}
%%%%%%%%%%%%%%%%%%%%%%%%%%%%%%%%%%%%%%%%%%%%%%%%%%%%%%%
\begin{frame}[fragile]\frametitle{Sintaxis}
    \begin{minted}{python}
    class ClaseHeredada(ClaseBase):
        def funcion(self):
            # [...]
    \end{minted}
\end{frame}
%%%%%%%%%%%%%%%%%%%%%%%%%%%%%%%%%%%%%%%%%%%%%%%%%%%%%%%
\begin{frame}[fragile]\frametitle{Ejemplo I}
    \begin{minted}{python}
class Bird:
  def intro(self):
    print("There are many types of birds.")
    
  def flight(self):
    print("Most of the birds can fly but some cannot.")
  
class sparrow(Bird):
  def flight(self):
    print("Sparrows can fly.")
    
class ostrich(Bird):
  def flight(self):
    print("Ostriches cannot fly.")
    
obj_bird = Bird()
obj_spr = sparrow()
obj_ost = ostrich()

obj_bird.intro()
obj_bird.flight()
    \end{minted}
\end{frame}
%%%%%%%%%%%%%%%%%%%%%%%%%%%%%%%%%%%%%%%%%%%%%%%%%%%%%%%
\begin{frame}[fragile]\frametitle{Ejemplo II}
    \begin{minted}{python}
class Animal:
    def speak(self):
        raise NotImplementedError("Subclass must implement this method")

class Dog(Animal):
    def speak(self):
        return "Woof!"

class Cat(Animal):
    def speak(self):
        return "Meow!"

# Create a list of Animal objects
animals = [Dog(), Cat()]

# Call the speak method on each object
for animal in animals:
    print(animal.speak())
    \end{minted}
\end{frame}
%%%%%%%%%%%%%%%%%%%%%%%%%%%%%%%%%%%%%%%%%%%%%%%%%%%%%%%
\begin{frame}\frametitle{Recap}
    \begin{itemize}
        \item Clases permiten crear abstracciones
        \item Todo objeto es una instancia de clase en Python
        \item Funciones mágicas permiten código legible!
    \end{itemize}
\end{frame}
%%%%%%%%%%%%%%%%%%%%%%%%%%%%%%%%%%%%%%%%%%%%%%%%%%%%%%%
\begin{frame}
    \maketitle
\end{frame}
%%%%%%%%%%%%%%%%%%%%%%%%%%%%%%%%%%%%%%%%%%%%%%%%%%%%%%%
\begin{frame}[noframenumbering]\frametitle{Mini ejercicios}
    \begin{itemize}
        \item Crear una clase Cuadrado que reciba como entrada el largo de sus aristas. Cree una función de clase que entregue el área del cuadrado.
        \item Extienda el ejemplo anterior al caso de un cubo.
    \end{itemize}
\end{frame}
%%%%%%%%%%%%%%%%%%%%%%%%%%%%%%%%%%%%%%%%%%%%%%%%%%%%%%%
\end{document}
