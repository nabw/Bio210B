\documentclass[14pt,aspectratio=169,xcolor=dvipsnames]{beamer}
\usetheme{SimplePlus}
\usepackage{booktabs}
\usepackage{minted}

\title[short title]{Clase 6: Clases y polimorfismo}
\subtitle{}
\author[NA Barnafi] {Nicolás Alejandro Barnafi Wittwer}
\institute[UC|CMM] 
{
    Pontificia Universidad Católica de Chile \\
    Centro de Modelamiento Matemático
}

\titlegraphic{
    \vspace{-1.8cm}
    \begin{flushright}
      \includegraphics[height=2.5cm]{../images/logos/puc.png} 
    \end{flushright}
}
\date{}

\begin{document}
%%%%%%%%%%%%%%%%%%%%%%%%%%%%%%%%%%%%%%%%%%%%%%%%%%%%%%%
\begin{frame}
    \maketitle
\end{frame}
%%%%%%%%%%%%%%%%%%%%%%%%%%%%%%%%%%%%%%%%%%%%%%%%%%%%%%%
%%%%%%%%%%%%%%%%%%%%%%%%%%%%%%%%%%%%%%%%%%%%%%%%%%%%%%%
\section{Clases y objetos}
%%%%%%%%%%%%%%%%%%%%%%%%%%%%%%%%%%%%%%%%%%%%%%%%%%%%%%%
%%%%%%%%%%%%%%%%%%%%%%%%%%%%%%%%%%%%%%%%%%%%%%%%%%%%%%%
\begin{frame}[fragile]\frametitle{Motivación}
    \begin{minted}{python}
>>> df = pd.DataFrame() # Constructor
>>> type(df)
  <class 'pandas.core.frame.DataFrame'>
    \end{minted}
    \begin{itemize}
        \item Crear nuestros propios tipos de datos
        \item Crear abstracciones 
    \[ \text{animal $\in$ ser vivo, cuadrado $\in$ figura }\]
        \item \emph{Programación Orientada a Objetos} (OOP)
    \end{itemize}
\end{frame}
%%%%%%%%%%%%%%%%%%%%%%%%%%%%%%%%%%%%%%%%%%%%%%%%%%%%%%%
\begin{frame}[fragile]\frametitle{Clases conocidas}
    \begin{minted}{python}
>>> type(pd.DataFrame()) # pandas
    <class 'pandas.core.frame.DataFrame'>
>>> type(np.linspace(0,1)) # numpy
    <class 'numpy.ndarray'>
>>> type(2) # base
    <class 'int'>
    \end{minted}
\end{frame}
%%%%%%%%%%%%%%%%%%%%%%%%%%%%%%%%%%%%%%%%%%%%%%%%%%%%%%%
\begin{frame}[fragile]\frametitle{Sintaxis}
    \begin{minted}{python}
class A:
   # Método constructor
   # self es obligatorio
   def __init__(self, n):
      self.n = n # son distintos!
      self.n2 = n**2
   def saludar(self): print("hola")

a = A(4) # Constructor
print(a.n, a.n2) # 4 16
print(type(a)) # <class '__main__.A'>
    \end{minted}

\end{frame}
%%%%%%%%%%%%%%%%%%%%%%%%%%%%%%%%%%%%%%%%%%%%%%%%%%%%%%%
\begin{frame}[fragile]\frametitle{Ejemplo: vector 2D}
    \begin{footnotesize}
    \begin{minted}{python}
class Vec2D:
   # Método constructor
   # self es obligatorio
   def __init__(self, x, y): # magic function
      self.x = x 
      self.y = y
   def __add__(self, v2): # magic function
      out = Vec2D(self.x, self.y) # copiar
      out.x += v2.x # agregar nuevos valores
      out.y += v2.y
      return out # retornar copia
   def __getitem__(self, i): # magic function
      if i == 0: return self.x
      if i == 1: return self.y
      else:
         print("ERROR") # i = 0 o 1
    \end{minted}
    \end{footnotesize}
\end{frame}
%%%%%%%%%%%%%%%%%%%%%%%%%%%%%%%%%%%%%%%%%%%%%%%%%%%%%%%
\darkSlide{Consola}
%%%%%%%%%%%%%%%%%%%%%%%%%%%%%%%%%%%%%%%%%%%%%%%%%%%%%%%
%%%%%%%%%%%%%%%%%%%%%%%%%%%%%%%%%%%%%%%%%%%%%%%%%%%%%%%
\section{Polimorfismo}
%%%%%%%%%%%%%%%%%%%%%%%%%%%%%%%%%%%%%%%%%%%%%%%%%%%%%%%
%%%%%%%%%%%%%%%%%%%%%%%%%%%%%%%%%%%%%%%%%%%%%%%%%%%%%%%
\begin{frame}\frametitle{Motivación}
    \begin{itemize}
        \item Abstraer "Figura"
    \[ \text{animal $\in$ ser vivo, cuadrado $\in$ figura }\]
        \item Sistema ecológico (pizarra)
    \end{itemize}
    $$ \text{Carnívoro}\quad\to\quad\text{Herbívoro}\quad\to\quad\text{Plantas} $$
\end{frame}
%%%%%%%%%%%%%%%%%%%%%%%%%%%%%%%%%%%%%%%%%%%%%%%%%%%%%%%
\begin{frame}[fragile]\frametitle{Sintaxis}
    \begin{minted}{python}
    class ClaseHeredada(ClaseBase):
        def funcion(self):
            # [...]
        # Opcional
        def __init__(self, params):
            super(ClaseHeredada, self).__init__(params)
    \end{minted}
\end{frame}
%%%%%%%%%%%%%%%%%%%%%%%%%%%%%%%%%%%%%%%%%%%%%%%%%%%%%%%
\begin{frame}[fragile]\frametitle{Ejemplo I (src: HowToGeek)}
    \begin{footnotesize}
    \begin{minted}{python}
class Bird:
  def intro(self): print("There are many types of birds.")
    
  def flight(self): print("Most of the birds can fly but some cannot.")
  
class sparrow(Bird):
  def flight(self): print("Sparrows can fly.")
    
class ostrich(Bird):
  def flight(self): print("Ostriches cannot fly.")
    
obj_bird = Bird(); obj_spr = sparrow(); obj_ost = ostrich()
obj_bird.flight(); obj_spr.flight(); obj_ost.flight()
    \end{minted}
    \end{footnotesize}
\end{frame}
%%%%%%%%%%%%%%%%%%%%%%%%%%%%%%%%%%%%%%%%%%%%%%%%%%%%%%%
\begin{frame}[fragile]\frametitle{Ejemplo II (src: HowToGeek)}
    \begin{footnotesize}
    \begin{minted}{python}
class Animal:
    def speak(self):
        raise NotImplementedError("Subclass must implement this method")

class Dog(Animal):
    def speak(self): return "Woof!"

class Cat(Animal):
    def speak(self): return "Meow!"

animals = [Dog(), Cat()]

for animal in animals:
    print(animal.speak())
    \end{minted}
    \end{footnotesize}
\end{frame}
%%%%%%%%%%%%%%%%%%%%%%%%%%%%%%%%%%%%%%%%%%%%%%%%%%%%%%%
\begin{frame}\frametitle{Recap}
    \begin{itemize}
        \item Clases permiten crear abstracciones
        \item Todo objeto es una instancia de clase en Python
        \item Funciones mágicas permiten código legible!
    \end{itemize}
\end{frame}
%%%%%%%%%%%%%%%%%%%%%%%%%%%%%%%%%%%%%%%%%%%%%%%%%%%%%%%
%%%%%%%%%%%%%%%%%%%%%%%%%%%%%%%%%%%%%%%%%%%%%%%%%%%%%%%
\begin{frame}
    \maketitle
\end{frame}
%%%%%%%%%%%%%%%%%%%%%%%%%%%%%%%%%%%%%%%%%%%%%%%%%%%%%%%
\begin{frame}[fragile,noframenumbering]\frametitle{Mini ejercicios}
    \begin{itemize}
        \item Crear una clase Cuadrado que reciba como entrada el largo de sus aristas. Cree una función de clase que entregue el área del cuadrado.
        \item Extienda el ejemplo anterior al caso de un cubo.
    \begin{footnotesize}
    \begin{minted}{python}
class Cuadrado:
   def __init__(self, l):
   def area(self):
    \end{minted}
    \end{footnotesize}

    \end{itemize}
\end{frame}
%%%%%%%%%%%%%%%%%%%%%%%%%%%%%%%%%%%%%%%%%%%%%%%%%%%%%%%
\end{document}
