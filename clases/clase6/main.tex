\documentclass[14pt,aspectratio=169,xcolor=dvipsnames]{beamer}
\usetheme{SimplePlus}
\usepackage{booktabs}
\usepackage{minted}

\title[short title]{Clase 6: Strings}
\subtitle{}
\author[NA Barnafi] {Nicolás Alejandro Barnafi Wittwer}
\institute[UC|CMM] 
{
    Pontificia Universidad Católica de Chile \\
    Centro de Modelamiento Matemático
}

\titlegraphic{
    \vspace{-1.8cm}
    \begin{flushright}
      \includegraphics[height=2.5cm]{../images/logos/puc.png} 
    \end{flushright}
}

\date{21/08/2024}
\setbeamercovered{transparent}

\begin{document}
%%%%%%%%%%%%%%%%%%%%%%%%%%%%%%%%%%%%%%%%%%%%%%%%%%%%%%%
\begin{frame}
    \maketitle
\end{frame}
%%%%%%%%%%%%%%%%%%%%%%%%%%%%%%%%%%%%%%%%%%%%%%%%%%%%%%%
\begin{frame}[t,fragile]\frametitle{Definición}
    \begin{itemize} 
        \item<1-> Un 'string' es un tipo de dato que contiene texto. 
            \begin{minted}{python}
        > s = "hola :)"
        > print(s) # Imprime 'hola :)'
            \end{minted}
        \item<2-> Está compuesto por caracteres, se pueden acceder por componentes: 
            \begin{minted}{python}
        > s = "hola :)"
        > print(s[0], s[1], s[2]) # Imprime 'hol'
            \end{minted}
        \item<3-> Tienen MUCHAS funcionalidades
    \end{itemize}
\end{frame}
%%%%%%%%%%%%%%%%%%%%%%%%%%%%%%%%%%%%%%%%%%%%%%%%%%%%%%%
\begin{frame}\frametitle{Functiones del objeto}
    Los strings, como muchos otros objetos, tienen funciones. Si tenemos \texttt{s = "hola string"}:
    \begin{itemize}
        \item \texttt{s.upper()}: Nuevo string 'HOLA STRING'
        \item \texttt{s.replace('str', 'k')}: Nuevo string 'hola king'
        \item \texttt{s.find('a')}: Entrega índice '3'.
        \item Hay MUCHAS (ver \texttt{dir(s)})
    \end{itemize}

En todas ellas, se general un \texttt{str} nuevo, el original queda intacto. Cuándo podría ser un problema esto? 
\end{frame}
%%%%%%%%%%%%%%%%%%%%%%%%%%%%%%%%%%%%%%%%%%%%%%%%%%%%%%%
\begin{frame}\frametitle{La librería \texttt{string}}
    Usaremos nuestra primera librería!
    \begin{itemize}
        \item Las librerías agregan funcionalidades específicas
        \item Sintaxis para importar librerías: 
            \begin{flushright} \texttt{> import string} \end{flushright}
        \item Esto genera una variable del nombre de la librería con funciones
    \end{itemize}

\vspace{1cm}
\pause\idea{Veamos cómo se puede usar}
\end{frame}
%%%%%%%%%%%%%%%%%%%%%%%%%%%%%%%%%%%%%%%%%%%%%%%%%%%%%%%
\begin{frame}[fragile]\frametitle{El método \texttt{format}}
    $$  \texttt{algo}\underbrace{.}_\text{llamar función}method() $$
Sirve para imprimir variables:
    \begin{minted}{python}
        > a = 20.17
        > "a es {}".format(a) # imprime 20.17
        > f"a es {a}" # imprime 20.17
        > f"a es {a:2.1f}" # imprime 20.1
        > f"a es {a:2.2e}" # imprime 2.02e+01
    \end{minted}
\idea{Método y método de tipo son diferentes}
\end{frame}
%%%%%%%%%%%%%%%%%%%%%%%%%%%%%%%%%%%%%%%%%%%%%%%%%%%%%%%
\begin{frame}\frametitle{Recap}
    \begin{itemize}
        \item Hoy vimos los strings
        \item Los strings son tipos de datos con funciones
        \item Aprendimos cómo importar la librería 'string'
        \item Aprendimos algunas funcionalidaddes de 'string'
    \end{itemize}
\end{frame}
%%%%%%%%%%%%%%%%%%%%%%%%%%%%%%%%%%%%%%%%%%%%%%%%%%%%%%%
\begin{frame}
    \maketitle
\end{frame}
\end{document}
