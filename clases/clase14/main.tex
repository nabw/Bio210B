\documentclass[14pt,aspectratio=169,xcolor=dvipsnames]{beamer}
\usetheme{SimplePlus}
\usepackage{booktabs}
\usepackage{minted}
\usepackage{mathtools}

\title[short title]{Clase 14: Python científico III, Pandas}
\subtitle{}
\author[NA Barnafi] {Nicolás Alejandro Barnafi Wittwer}
\institute[UC|CMM] 
{
    Pontificia Universidad Católica de Chile \\
    Centro de Modelamiento Matemático
}

\titlegraphic{
    \vspace{-1.8cm}
    \begin{flushright}
      \includegraphics[height=2.5cm]{../images/logos/puc.png} 
    \end{flushright}
}

\date{11/09/2024}
%\setbeamercovered{transparent}

\begin{document}
%%%%%%%%%%%%%%%%%%%%%%%%%%%%%%%%%%%%%%%%%%%%%%%%%%%%%%%
\begin{frame}
    \maketitle
\end{frame}
%%%%%%%%%%%%%%%%%%%%%%%%%%%%%%%%%%%%%%%%%%%%%%%%%%%%%%%
\begin{frame}\frametitle{Temas}
    \begin{itemize}
        \item Pandas
    \end{itemize}

\end{frame}
%%%%%%%%%%%%%%%%%%%%%%%%%%%%%%%%%%%%%%%%%%%%%%%%%%%%%%%
\begin{frame}[fragile]\frametitle{Tipo central: \code{DataFrame}}
    \begin{minted}{python}
  import pandas
  mydataset = {
    'cars': ["BMW", "Volvo", "Ford"],
    'passings': [3, 7, 2]
  }
  myvar = pandas.DataFrame(mydataset)
  print(myvar) 
    \end{minted}

    \begin{minted}{bash}
  $      cars    passings
    0    BMW         3
    1  Volvo         7
    2   Ford         2
    \end{minted}
\end{frame}
%%%%%%%%%%%%%%%%%%%%%%%%%%%%%%%%%%%%%%%%%%%%%%%%%%%%%%%
\begin{frame}\frametitle{\code{DataFrame}}
    \begin{itemize}
        \item Es el elemento central de Pandas
        \item Son \emph{tablas de datos}
        \item Se pueden filtrar fácilmente
    \end{itemize}
\end{frame}
%%%%%%%%%%%%%%%%%%%%%%%%%%%%%%%%%%%%%%%%%%%%%%%%%%%%%%%
\begin{frame}[fragile]\frametitle{Nombre de datos}

    \begin{minted}{python}
  data = {
    "calories": [420, 380, 390],
    "duration": [50, 40, 45] }

  df = pd.DataFrame(data)
  print(df) 
  df = pd.DataFrame(data, index = ["day1", "day2", "day3"])
  print(df) 
    \end{minted}
\end{frame}
%%%%%%%%%%%%%%%%%%%%%%%%%%%%%%%%%%%%%%%%%%%%%%%%%%%%%%%
\begin{frame}[fragile]\frametitle{Cargar datos}

    Asumiendo que existe archivo \code{data.csv}
    \begin{minted}{python}
  import pandas as pd

  df = pd.read_csv('data.csv')
  print(df)
    \end{minted}

\end{frame}
%%%%%%%%%%%%%%%%%%%%%%%%%%%%%%%%%%%%%%%%%%%%%%%%%%%%%%%
\begin{frame}[fragile]\frametitle{Explorando datos}
    \begin{minted}{python}
  df = pd.read_csv('data.csv')
    \end{minted}
    \begin{itemize}
        \item \code{df.head(n)}: Primeros \code{n} datos
        \item \code{df.tail(n)}: Últimos \code{n} datos
        \item \code{df.describe()}: Ver estadísticas descriptivas
        \item \code{df.sort\_values(by="A")}: Ordenar según columna "A"
    \end{itemize}
    
    \idea{Consola}
\end{frame}
%%%%%%%%%%%%%%%%%%%%%%%%%%%%%%%%%%%%%%%%%%%%%%%%%%%%%%%
\begin{frame}[fragile]\frametitle{Acceso a datos}
    \begin{minted}{python}
  df = pd.read_csv('data.csv')
    \end{minted}
    \begin{itemize}
        \item \code{df["A"]}: Extraer columna con etiqueta "A"
        \item \code{df[1:4]}: Extraer filas en [1,4)
        \item \code{df[1:4, ["A", "B"]]}: Filas y columnas
    \end{itemize}

    \idea{Consola}
\end{frame}
%%%%%%%%%%%%%%%%%%%%%%%%%%%%%%%%%%%%%%%%%%%%%%%%%%%%%%%
\begin{frame}[fragile]\frametitle{Filtros booleanos}

    Exploremos los siguientes comandos: 
    \begin{minted}{python}
    df = pd.read_csv('data.csv')
    df["Pulse"] > 120
    df[df["Pulse"] > 120]
    \end{minted}

    \vspace{1cm}
    \pause Se llama filtro por máscara, o máscara booleana.
\end{frame}
%%%%%%%%%%%%%%%%%%%%%%%%%%%%%%%%%%%%%%%%%%%%%%%%%%%%%%%
\begin{frame}\frametitle{Recap}
    \begin{itemize}
        \item Pandas sirve para procesar datos
        \item También se puede hacer estadística (no lo vimos)
        \item Máscaras booleanas
    \end{itemize}

    \pause
    \vspace{1cm}
    \code{https://pandas.pydata.org/docs/user\_guide/10min.html}
\end{frame}
%%%%%%%%%%%%%%%%%%%%%%%%%%%%%%%%%%%%%%%%%%%%%%%%%%%%%%%
\begin{frame}
    \maketitle
\end{frame}
%%%%%%%%%%%%%%%%%%%%%%%%%%%%%%%%%%%%%%%%%%%%%%%%%%%%%%%
\begin{frame}[fragile,noframenumbering]\frametitle{Mini ejercicios}
    \begin{itemize}
        \item Genere una sucesión de 0's y 1's que representan lanzamientos de monedas. Guárdelo en un archivo \code{.csv} y cárguelo en Pandas.
        \item Use las funciones de pandas para encontrar la probabilidad, en su muestra, de obtener una cara y la de obtener un sello.
    \end{itemize}
\end{frame}
%%%%%%%%%%%%%%%%%%%%%%%%%%%%%%%%%%%%%%%%%%%%%%%%%%%%%%%
\end{document}
