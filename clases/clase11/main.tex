\documentclass[14pt,aspectratio=169,xcolor=dvipsnames]{beamer}
\usetheme{SimplePlus}
\usepackage{booktabs}
\usepackage{minted}

\title[short title]{Clase 11: Módulos}
\subtitle{}
\author[NA Barnafi] {Nicolás Alejandro Barnafi Wittwer}
\institute[UC|CMM] 
{
    Pontificia Universidad Católica de Chile \\
    Centro de Modelamiento Matemático
}

\titlegraphic{
    \vspace{-1.8cm}
    \begin{flushright}
      \includegraphics[height=2.5cm]{../images/logos/puc.png} 
    \end{flushright}
}

\date{11/09/2024}
%\setbeamercovered{transparent}

\begin{document}
%%%%%%%%%%%%%%%%%%%%%%%%%%%%%%%%%%%%%%%%%%%%%%%%%%%%%%%
\begin{frame}
    \maketitle
\end{frame}
%%%%%%%%%%%%%%%%%%%%%%%%%%%%%%%%%%%%%%%%%%%%%%%%%%%%%%%
\begin{frame}[fragile]\frametitle{Motivación}
    Ya conocemos módulos: \code{string}, \code{sys}
    \begin{itemize}
        \item \code{import string}
        \item \code{import sys}
    \end{itemize}
    Así se cargan ciertas funcionalidades. 

    \pause Hoy aprenderemos a crear nuevos
\end{frame}
%%%%%%%%%%%%%%%%%%%%%%%%%%%%%%%%%%%%%%%%%%%%%%%%%%%%%%%
\begin{frame}[fragile]\frametitle{Pip}
    \textbf{P}ackage \textbf{I}nstaller for \textbf{P}ython
    
    \begin{itemize}
        \item Gestor de paquetes que viene instalado junto a Python
        \item Info online en \code{https://pypi.org/}
        \item Fácil de usar por Bash
            \begin{minted}{bash}
        $ pip install pandas
        $ python
        > import pandas  # recién instalada!
            \end{minted}
    \end{itemize}
    
    \pause Dónde están los programas...?
\end{frame}
%%%%%%%%%%%%%%%%%%%%%%%%%%%%%%%%%%%%%%%%%%%%%%%%%%%%%%%
\begin{frame}\frametitle{el PATH}
    Cuando escribo \code{ls} en Bash... de dónde sale? \\

    \vspace{0.5cm}
    La variable de entorno PATH tiene las carpetas donde buscar programas\footnote{Se puede modificar para tener programas en nuevas carpetas, i.e. \code{/opt}}

    \pause \idea{Consola...}

    
\end{frame}
%%%%%%%%%%%%%%%%%%%%%%%%%%%%%%%%%%%%%%%%%%%%%%%%%%%%%%%
\begin{frame}[fragile]\frametitle{Crear módulo}
    \code{operaciones.py}
    \begin{minted}{python}
  def suma(x,y): return x+y

  def mult(x,y): return x*y
    \end{minted}

  En Python:
  \begin{minted}{python}
  > import operaciones # Debe estar en el path
  > # from operaciones import suma,prod
  > operaciones.suma(2,3) # 5
  > operaciones.mult(2,3) # 6
  \end{minted}
\end{frame}
%%%%%%%%%%%%%%%%%%%%%%%%%%%%%%%%%%%%%%%%%%%%%%%%%%%%%%%
\begin{frame}[fragile]\frametitle{el PATH en Python}
    \begin{minted}{bash}
  ~/clases/Bio210B$ python -m site
   sys.path = [
       '/home/nico/clases/Bio210B',
       '/usr/lib/python3/dist-packages' # son más...
    \end{minted}

    \pause Se puede extender con PYTHONPATH
    \begin{minted}{bash}
  $ export PYTHONPATH=/home/nico/otraCarpeta
  $ python -m site
   sys.path = [
       '/home/nico/clases/Bio210B',
       '/home/nico/otraCarpeta',
       '/usr/lib/python3/dist-packages'
    \end{minted}

\end{frame}
%%%%%%%%%%%%%%%%%%%%%%%%%%%%%%%%%%%%%%%%%%%%%%%%%%%%%%%
\begin{frame}[fragile]\frametitle{Nota: variables en Bash}
    \begin{minted}{bash}
  $ export PYTHONPATH=[...]
    \end{minted}
    genera una variable de ambiente. Se puede ver con
    \begin{minted}{bash}
  $ echo ${PYTHONPATH}
    \end{minted}
  Todas las definidas se ven con programa \code{env}
\end{frame}
%%%%%%%%%%%%%%%%%%%%%%%%%%%%%%%%%%%%%%%%%%%%%%%%%%%%%%%
\begin{frame}\frametitle{Recap}
    \begin{itemize}
        \item Dónde están los módulos por defecto
        \item Cómo agregar un módulo
        \item Cómo hacer módulo detectable con PYTHONPATH
    \end{itemize}
\end{frame}
%%%%%%%%%%%%%%%%%%%%%%%%%%%%%%%%%%%%%%%%%%%%%%%%%%%%%%%
\begin{frame}
    \maketitle
\end{frame}
%%%%%%%%%%%%%%%%%%%%%%%%%%%%%%%%%%%%%%%%%%%%%%%%%%%%%%%
\begin{frame}[fragile]\frametitle{Mini ejercicios}
    \code{operaciones.py}
    \begin{minted}{python}
  def suma(x,y): return x+y

  def mult(x,y): return x*y
    \end{minted}

    \begin{itemize}
        \item Crear el módulo \code{operaciones} e importarlo desde un script.
        \item Guardar el módulo en otro lado e importarlo usando \code{PYTHONPATH}
    \end{itemize}

\end{frame}
%%%%%%%%%%%%%%%%%%%%%%%%%%%%%%%%%%%%%%%%%%%%%%%%%%%%%%%
\end{document}
