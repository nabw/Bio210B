\documentclass[14pt,aspectratio=169,xcolor=dvipsnames]{beamer}
\usetheme{SimplePlus}
\usepackage{booktabs}
\usepackage{minted}

\title[short title]{Clase 20: Introducción a R II}
\subtitle{}
\author[NA Barnafi] {Nicolás Alejandro Barnafi Wittwer}
\institute[UC|CMM] 
{
    Pontificia Universidad Católica de Chile \\
    Centro de Modelamiento Matemático
}

\titlegraphic{
    \vspace{-1.8cm}
    \begin{flushright}
      \includegraphics[height=2.5cm]{../images/logos/puc.png} 
    \end{flushright}
}

\date{16/10/2024}
%\setbeamercovered{transparent}

\begin{document}
%%%%%%%%%%%%%%%%%%%%%%%%%%%%%%%%%%%%%%%%%%%%%%%%%%%%%%%
\begin{frame}
    \maketitle
\end{frame}
%%%%%%%%%%%%%%%%%%%%%%%%%%%%%%%%%%%%%%%%%%%%%%%%%%%%%%%
\begin{frame}\frametitle{Motivación}
    \begin{itemize}
        \item R fue \emph{hecho por} estadísticos
        \item Tiene funcionalidades incluídas
        \item Tiene bases de datos por defecto
    \end{itemize}
    
    \code{mtcars,iris,datasets::[...], ?iris, ?mtcars}
\end{frame}
%%%%%%%%%%%%%%%%%%%%%%%%%%%%%%%%%%%%%%%%%%%%%%%%%%%%%%%
\begin{frame}[fragile]\frametitle{Exploring data}
    \begin{minted}{R}
Data.Cars <- mtcars 
dim(Data.Cars)
names(Data.Cars) 
rownames(Data.Cars)
Data.Cars$cyl  # Comparar con Data.Cars["cyl"]
summary(Data.Cars)
max(iris["Petal.Length"]) 
which.max(iris$Petal.Length) # con [] no funciona
mean(iris$Petal.Length)
    \end{minted}
\end{frame}
%%%%%%%%%%%%%%%%%%%%%%%%%%%%%%%%%%%%%%%%%%%%%%%%%%%%%%%
\begin{frame}[fragile]\frametitle{Indexaciones y cálculos}
Si 'which' solo opera sobre vectores, cómo extraigo por nombre? \pause

    \begin{minted}{R}  
    datos = iris$"Petal.Length"
    filas = rownames(iris)
    indx_max = which.max(datos)
    filas[indx_max]
    \end{minted}
\end{frame}
%%%%%%%%%%%%%%%%%%%%%%%%%%%%%%%%%%%%%%%%%%%%%%%%%%%%%%%
\begin{frame}[fragile]\frametitle{Estadística}
    \begin{small}
\begin{minted}{R}
  model = lm(Petal.Length ~ Sepal.Length, data = iris)
  summary(model)
  b = model$coefficients[1]
  m = model$coefficients[2]
  minx = min(iris$Sepal.Length)
  maxx = max(iris$Sepal.Length)
  plot(c(minx, maxx), c(b+minx*m, b+maxx*m), 
      type="l", col="red") # (l)ine
  lines(iris$Sepal.Length, iris$Petal.Length, 
        type="p", col="blue") # (p)oint
\end{minted}
    \end{small}
    \idea{Consola...}
\end{frame}
%%%%%%%%%%%%%%%%%%%%%%%%%%%%%%%%%%%%%%%%%%%%%%%%%%%%%%%
\begin{frame}
    \maketitle
\end{frame}
%%%%%%%%%%%%%%%%%%%%%%%%%%%%%%%%%%%%%%%%%%%%%%%%%%%%%%%
\begin{frame}[noframenumbering, fragile]\frametitle{Mini ejercicios}
    \begin{itemize}
        \item Explorar bases de datos de R
        \item En una base de datos elegida, elegir variables para hacer modelos lineales
        \item Qué puede inferir de \code{summary(modelo)}?
        \item Se pueden agregar funciones no lineales de parámetros:
            \begin{minted}{R}
    lm(y ~ x + I(x^3) + log(x), data=datos)
            \end{minted}
    \end{itemize}
\end{frame}
%%%%%%%%%%%%%%%%%%%%%%%%%%%%%%%%%%%%%%%%%%%%%%%%%%%%%%%
\end{document}
