\documentclass[14pt,aspectratio=169,xcolor=dvipsnames]{beamer}
\usetheme{SimplePlus}
\usepackage{booktabs}
\usepackage{minted}

\title[short title]{Clase 18: git y desarrollo colaborativo}
\subtitle{}
\author[NA Barnafi] {Nicolás Alejandro Barnafi Wittwer}
\institute[UC|CMM] 
{
    Pontificia Universidad Católica de Chile \\
    Centro de Modelamiento Matemático
}

\titlegraphic{
    \vspace{-1.8cm}
    \begin{flushright}
      \includegraphics[height=2.5cm]{../images/logos/puc.png} 
    \end{flushright}
}

\date{16/10/2024}
%\setbeamercovered{transparent}

\begin{document}
%%%%%%%%%%%%%%%%%%%%%%%%%%%%%%%%%%%%%%%%%%%%%%%%%%%%%%%
\begin{frame}
    \maketitle
\end{frame}
%%%%%%%%%%%%%%%%%%%%%%%%%%%%%%%%%%%%%%%%%%%%%%%%%%%%%%%
\begin{frame}\frametitle{Motivación}
    {\huge
    \begin{center}
        \alertBlue{Cómo compartir código?}
    \end{center}
    }
    \begin{flushright}
    \includegraphics[height=2cm]{../images/logos/git.png}
    \end{flushright}
\end{frame}
%%%%%%%%%%%%%%%%%%%%%%%%%%%%%%%%%%%%%%%%%%%%%%%%%%%%%%%
\begin{frame}[t,fragile]\frametitle{Cómo funciona git}
  \begin{columns}
    \begin{column}[t]{0.45\textwidth}
        Original
        \begin{minted}{python}
def suma(x,y):
    return x+y
        \end{minted}
    \end{column}
    \begin{column}[t]{0.45\textwidth}
        Modificado
        \begin{minted}{python}
def suma(x,y):
    """
    Función que suma
    """
    return x+y
        \end{minted}
    \end{column}
  \end{columns}

  \vspace{1cm}
  \begin{itemize}
  \item Diferencia: 3 líneas agregadas en la línea 2. 

  \item Se guardan solo los \alert{incrementos} de archivos
  \end{itemize}
\end{frame}
%%%%%%%%%%%%%%%%%%%%%%%%%%%%%%%%%%%%%%%%%%%%%%%%%%%%%%%
\begin{frame}\frametitle{Función general (offline)}
    Flujo de trabajo:
    \begin{itemize}
        \item Cambios propuestos de archivos \phantom{ }
            
            \begin{minipage}{0.35\textwidth}
            \begin{block}{}
                Stage \hfill(\code{git add})
            \end{block}
            \end{minipage}
        \item Deshacer cambios propuestos \phantom{ }
            
            \begin{minipage}{0.35\textwidth}
            \begin{block}{}
                Stage \hfill(\code{git reset})
            \end{block}
            \end{minipage}

        \item Cambios incorporados \phantom{ }

            \begin{minipage}{0.35\textwidth}
            \begin{block}{}
                Commit \hfill(\code{git commit})
            \end{block}
            \end{minipage}
    \end{itemize}
\end{frame}
%%%%%%%%%%%%%%%%%%%%%%%%%%%%%%%%%%%%%%%%%%%%%%%%%%%%%%%
\begin{frame}[t,fragile]\frametitle{Ejemplo :)}
  \begin{columns}
    \begin{column}[t]{0.45\textwidth}
        Original
        \begin{minted}{python}
def suma(x,y):
    return x+y
        \end{minted}
    \end{column}
    \begin{column}[t]{0.45\textwidth}
        Modificado
        \begin{minted}{python}
def suma(x,y):
    """
    Función que suma
    """
    return x+y
        \end{minted}
    \end{column}
  \end{columns}
    
\begin{enumerate}
    \item Tengo archivo original $\to$ lo modifico
    \item Agrego cambios como nueva modificación propuesta \hfill (\alertGreen{stage})
    \item Incorporo modificación \hfill (\alertGreen{commit})
\end{enumerate}
\idea{Consola}
\end{frame}
%%%%%%%%%%%%%%%%%%%%%%%%%%%%%%%%%%%%%%%%%%%%%%%%%%%%%%%%
\begin{frame}\frametitle{Función general (online)}
    Flujo de trabajo:
    \begin{itemize}
        \item Crear carpeta con \emph{copia local} de repositorio online
            \begin{minipage}{0.35\textwidth}
            \begin{block}{}
                Clone \hfill(\code{git clone})
            \end{block}
            \end{minipage}

        \item Descargar contenidos de repositorio online \phantom{ }
            
            \begin{minipage}{0.35\textwidth}
            \begin{block}{}
                Pull \hfill(\code{git pull})
            \end{block}
            \end{minipage}
        \item Enviar cambios creados \phantom{ }

            \begin{minipage}{0.35\textwidth}
            \begin{block}{}
                Commit \hfill(\code{git push})
            \end{block}
            \end{minipage}
    \end{itemize}
\end{frame}
%%%%%%%%%%%%%%%%%%%%%%%%%%%%%%%%%%%%%%%%%%%%%%%%%%%%%%%
\begin{frame}\frametitle{Otras funcionalidades}
    \begin{itemize}
        \item Manejo de conflictos 
        \item Crear ramas para desarrollo paralelo (\code{git branch})
        \item Deshacer cambios (\code{git reset})
        \item Ir a versiones anteriores del código (\code{git checkout})
        \item Manejo de protocolos de seguridad (SSH)
        \item ... y \alert{muchas} más
    \end{itemize}
\end{frame}
%%%%%%%%%%%%%%%%%%%%%%%%%%%%%%%%%%%%%%%%%%%%%%%%%%%%%%%
\begin{frame}\frametitle{Recap}
    \begin{itemize}
        \item \code{git} es LA herramienta de desarrollo colaborativo
        \item Se basa en un grafo de estados
        \item Modificaciones locales: \code{add}, \code{commit}, \code{reset}
        \item Modificaciones online: \code{clone}, \code{pull}, \code{push}
    \end{itemize}
\end{frame}
%%%%%%%%%%%%%%%%%%%%%%%%%%%%%%%%%%%%%%%%%%%%%%%%%%%%%%%
\begin{frame}
    \maketitle
\end{frame}
%%%%%%%%%%%%%%%%%%%%%%%%%%%%%%%%%%%%%%%%%%%%%%%%%%%%%%%
\begin{frame}[noframenumbering]\frametitle{Mini ejercicios}
    \begin{enumerate}
        \item Instalar git
        \item Crear cuenta en \code{github.com}
        \item Crear repositorio en su cuenta
        \item Clonar el repositorio localmente
        \item Agregar un cambio
        \item Hacer \emph{push} del cambio y verificar el cambio
    \end{enumerate}
    
    \begin{center}
        Turorial online: \code{https://www.w3schools.com/git/}
    \end{center}
\end{frame}
%%%%%%%%%%%%%%%%%%%%%%%%%%%%%%%%%%%%%%%%%%%%%%%%%%%%%%%
\end{document}
