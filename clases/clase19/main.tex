\documentclass[14pt,aspectratio=169,xcolor=dvipsnames]{beamer}
\usetheme{SimplePlus}
\usepackage{booktabs}
\usepackage{minted}

\title[short title]{Clase 19: Introducción a R}
\subtitle{}
\author[NA Barnafi] {Nicolás Alejandro Barnafi Wittwer}
\institute[UC|CMM] 
{
    Pontificia Universidad Católica de Chile \\
    Centro de Modelamiento Matemático
}

\titlegraphic{
    \vspace{-1.8cm}
    \begin{flushright}
      \includegraphics[height=2.5cm]{../images/logos/puc.png} 
    \end{flushright}
}

\date{16/10/2024}
%\setbeamercovered{transparent}

\begin{document}
%%%%%%%%%%%%%%%%%%%%%%%%%%%%%%%%%%%%%%%%%%%%%%%%%%%%%%%
\begin{frame}
    \maketitle
\end{frame}
%%%%%%%%%%%%%%%%%%%%%%%%%%%%%%%%%%%%%%%%%%%%%%%%%%%%%%%
\begin{frame}\frametitle{Motivación}
    \begin{itemize}
        \item Lenguaje especializado en estadística
        \item Manejo de datos tipo Pandas (en realidad al revés)
        \item Muy bueno para visualización de datos
        \item Interpretado, sin objetos (clases)
    \end{itemize}
    \begin{flushright}
        \includegraphics[height=2cm]{../images/logos/R.png}
    \end{flushright}
\end{frame}
%%%%%%%%%%%%%%%%%%%%%%%%%%%%%%%%%%%%%%%%%%%%%%%%%%%%%%%
\begin{frame}[fragile]\frametitle{Ejecución}
    \begin{itemize}
        \item Al igual que Python, consola y programa
        \item Consola
            \begin{minted}{bash}
    $ R         # Abrir programa en bash
    > 2 + 2     # Ejecutar en consola de R
    [1] 4       # Resultado 
    > q()       # Salir de consola
            \end{minted}
        \item Programa
            \begin{minted}{bash}
    $ Rscript test.r   # Ejecutar script con '2+2'
    [1] 4              # Resultado 
            \end{minted}

    \end{itemize}
\end{frame}
%%%%%%%%%%%%%%%%%%%%%%%%%%%%%%%%%%%%%%%%%%%%%%%%%%%%%%%
\begin{frame}[fragile]\frametitle{Sintaxis: Tipos}
    \begin{itemize}
        \item String: \code{"Hola"}
        \item Números: \code{1,  3.5}
        \item Comentario: \code{"hola" \# string que saluda}
        \item Imprimir: \code{print("hola")}
        \item Asignar variable: \code{a=2, a<-2, 2->a}
        \item Variables pueden tener puntos: \code{.a=2}
        \item Sumar strings da error!
                
            \begin{flushright} \code{paste("hola","chao")}\end{flushright}
        \item Números son "numeric" por defecto (ver con \code{class})
        \item Booleanos: \code{TRUE},\code{FALSE}
    \end{itemize}
\end{frame}
%%%%%%%%%%%%%%%%%%%%%%%%%%%%%%%%%%%%%%%%%%%%%%%%%%%%%%%
\begin{frame}\frametitle{Sintaxis: Tipos II}
    \begin{itemize}
        \item Tienen nombres específicos:
            \begin{itemize}
                \item numeric \code{4,5.5,-2}
                \item integer \code{1L, -3L, 100L}  (si, con la 'L')
                \item complex \code{9+3i}
                \item character  (strings)
                \item logical   (booleanos)
            \end{itemize}
        \item Convertir números: \code{as.numeric()|as.integer()|as.complex()}
    \end{itemize}
\end{frame}
%%%%%%%%%%%%%%%%%%%%%%%%%%%%%%%%%%%%%%%%%%%%%%%%%%%%%%%
\begin{frame}\frametitle{Sintaxis II: Operaciones}
    \begin{itemize}
        \item Booleanos: \code{a\&b} (and), \code{a|b} (or), \code{!a} (not)
        \item Comparación: \code{a<b}, \code{a<=b}, \code{3==3}, \code{3!=2}
        \item Números: \code{2+2}, \code{2*3}, \code{2**3==2\^3}, \code{10\%\%3}
    \end{itemize}
\end{frame}
%%%%%%%%%%%%%%%%%%%%%%%%%%%%%%%%%%%%%%%%%%%%%%%%%%%%%%%
\begin{frame}[fragile]\frametitle{Sintaxis III: if/else}
    \begin{minted}{R}
    a = 2
    b = 3
    if (a>b) {
        print("a>b")
    }
    else if (a==b) {
        print("a==b")
    }
    else { # a<b
        print("a<b")
    }
    \end{minted}
\end{frame}
%%%%%%%%%%%%%%%%%%%%%%%%%%%%%%%%%%%%%%%%%%%%%%%%%%%%%%%
\begin{frame}[fragile]\frametitle{Sintaxis IV: while}
\end{frame}
%%%%%%%%%%%%%%%%%%%%%%%%%%%%%%%%%%%%%%%%%%%%%%%%%%%%%%%
\begin{frame}[fragile]\frametitle{Sintaxis V: for}
\end{frame}
%%%%%%%%%%%%%%%%%%%%%%%%%%%%%%%%%%%%%%%%%%%%%%%%%%%%%%%
\begin{frame}[fragile]\frametitle{Sintaxis VI: functions}
\end{frame}
%%%%%%%%%%%%%%%%%%%%%%%%%%%%%%%%%%%%%%%%%%%%%%%%%%%%%%%
\begin{frame}
    \maketitle
\end{frame}
%%%%%%%%%%%%%%%%%%%%%%%%%%%%%%%%%%%%%%%%%%%%%%%%%%%%%%%
\begin{frame}[noframenumbering]\frametitle{Mini ejercicios}
\end{frame}
%%%%%%%%%%%%%%%%%%%%%%%%%%%%%%%%%%%%%%%%%%%%%%%%%%%%%%%
\end{document}
