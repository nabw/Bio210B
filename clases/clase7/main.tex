\documentclass[14pt,aspectratio=169,xcolor=dvipsnames]{beamer}
\usetheme{SimplePlus}
\usepackage{booktabs}
\usepackage{minted}

\title[short title]{Clase 7: Estructuras de datos}
\subtitle{}
\author[NA Barnafi] {Nicolás Alejandro Barnafi Wittwer}
\institute[UC|CMM] 
{
    Pontificia Universidad Católica de Chile \\
    Centro de Modelamiento Matemático
}

\titlegraphic{
    \vspace{-1.8cm}
    \begin{flushright}
      \includegraphics[height=2.5cm]{../images/logos/puc.png} 
    \end{flushright}
}

%\date{26/07/2024, Vancouver, WCCM}
\setbeamercovered{transparent}

\begin{document}
%%%%%%%%%%%%%%%%%%%%%%%%%%%%%%%%%%%%%%%%%%%%%%%%%%%%%%%
\begin{frame}
    \maketitle
\end{frame}
%%%%%%%%%%%%%%%%%%%%%%%%%%%%%%%%%%%%%%%%%%%%%%%%%%%%%%%
\begin{frame}[fragile]\frametitle{Motivación}
En el ejemplo del IMC, se vuelve impracticable tener muchos datos:
    \begin{minted}{python}
        peso1 = ...
        altura1 = ...
        peso2 = ... 
        altura2 = ...
        ...
        peso1200 = ...
        altura1200 = ...
    \end{minted}

Nos gustaría tener una manera de manejar cantidades \emph{arbitrarias} de datos y operar sobre ellas.
\end{frame}
%%%%%%%%%%%%%%%%%%%%%%%%%%%%%%%%%%%%%%%%%%%%%%%%%%%%%%%
\begin{frame}[fragile]\frametitle{Un loop de iteración}
    \begin{minted}{python}
      lista = [1,2,3,4,5]
      for e in lista:
        print(e)
    \end{minted}
\pause El resultado será

1
2
3
4
5
\end{frame}
%%%%%%%%%%%%%%%%%%%%%%%%%%%%%%%%%%%%%%%%%%%%%%%%%%%%%%%
\begin{frame}\frametitle{Un poco de sintaxis}
    $$ \texttt{a = [1,2,3,4,5,8]} $$
    \begin{itemize}
        \item asdf...
    \end{itemize}
\end{frame}
%%%%%%%%%%%%%%%%%%%%%%%%%%%%%%%%%%%%%%%%%%%%%%%%%%%%%%%
\begin{frame}
    \maketitle
\end{frame}
\end{document}
