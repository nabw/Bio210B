\documentclass[14pt,aspectratio=169,xcolor=dvipsnames]{beamer}
\usetheme{SimplePlus}
\usepackage{booktabs}
\usepackage{minted}

\title[short title]{Clase 9: Introducción a R}
\subtitle{}
\author[NA Barnafi] {Nicolás Alejandro Barnafi Wittwer}
\institute[UC|CMM] 
{
    Pontificia Universidad Católica de Chile \\
    Centro de Modelamiento Matemático
}

\titlegraphic{
    \vspace{-1.8cm}
    \begin{flushright}
      \includegraphics[height=2.5cm]{../images/logos/puc.png} 
    \end{flushright}
}
\date{}

\begin{document}
%%%%%%%%%%%%%%%%%%%%%%%%%%%%%%%%%%%%%%%%%%%%%%%%%%%%%%%
\begin{frame}
    \maketitle
\end{frame}
%%%%%%%%%%%%%%%%%%%%%%%%%%%%%%%%%%%%%%%%%%%%%%%%%%%%%%%
\begin{frame}\frametitle{Motivación}
    \begin{itemize}
        \item Lenguaje especializado en estadística
        \item Manejo de datos tipo Pandas (en realidad al revés)
        \item Muy bueno para visualización de datos
        \item Interpretado, sin objetos (clases)
    \end{itemize}
    \begin{flushright}
        \includegraphics[height=2cm]{../images/logos/R.png}
    \end{flushright}
\end{frame}
%%%%%%%%%%%%%%%%%%%%%%%%%%%%%%%%%%%%%%%%%%%%%%%%%%%%%%%
\begin{frame}[fragile]\frametitle{Ejecución}
    \begin{itemize}
        \item Al igual que Python, consola y programa
        \item Consola
            \begin{minted}{bash}
    $ R         # Abrir programa en bash
    > 2 + 2     # Ejecutar en consola de R
    [1] 4       # Resultado 
    > q()       # Salir de consola
            \end{minted}
        \item Programa
            \begin{minted}{bash}
    $ Rscript test.r   # Ejecutar script con '2+2'
    [1] 4              # Resultado 
            \end{minted}

    \end{itemize}
\end{frame}
%%%%%%%%%%%%%%%%%%%%%%%%%%%%%%%%%%%%%%%%%%%%%%%%%%%%%%%
\begin{frame}[fragile]\frametitle{Sintaxis: Tipos}
    \begin{itemize}
        \item String: \code{"Hola"}
        \item Números: \code{1,  3.5}
        \item Comentario: \code{"hola" \# string que saluda}
        \item Imprimir: \code{print("hola")}
        \item Asignar variable: \code{a=2, a<-2, 2->a}
        \item Variables pueden tener puntos: \code{.a=2}
        \item Sumar strings da error!
                
            \begin{flushright} \code{paste("hola","chao")}\end{flushright}
        \item Números son "numeric" por defecto (ver con \code{class})
        \item Booleanos: \code{TRUE},\code{FALSE}
    \end{itemize}
\end{frame}
%%%%%%%%%%%%%%%%%%%%%%%%%%%%%%%%%%%%%%%%%%%%%%%%%%%%%%%
\begin{frame}\frametitle{Sintaxis: Tipos II}
    \begin{itemize}
        \item Tienen nombres específicos:
            \begin{itemize}
                \item numeric \code{4,5.5,-2}
                \item integer \code{1L, -3L, 100L}  (si, con la 'L')
                \item complex \code{9+3i}
                \item character  (strings)
                \item logical   (booleanos)
            \end{itemize}
        \item Convertir números: \code{as.numeric()|as.integer()|as.complex()}
    \end{itemize}
\end{frame}
%%%%%%%%%%%%%%%%%%%%%%%%%%%%%%%%%%%%%%%%%%%%%%%%%%%%%%%
\begin{frame}\frametitle{Sintaxis II: Operaciones}
    \begin{itemize}
        \item Booleanos: \code{a\&b} (and), \code{a|b} (or), \code{!a} (not)
        \item Comparación: \code{a<b}, \code{a<=b}, \code{3==3}, \code{3!=2}
        \item Números: \code{2+2}, \code{2*3}, \code{2**3==2\^{}3}, \code{10\%\%3}
    \end{itemize}
\end{frame}
%%%%%%%%%%%%%%%%%%%%%%%%%%%%%%%%%%%%%%%%%%%%%%%%%%%%%%%
\begin{frame}[fragile]\frametitle{Sintaxis III: if/else}
    \begin{minted}{R}
    a = 2
    b = 3
    if (a>b) {
        print("a>b")
    }
    else if (a==b) {
        print("a==b")
    }
    else { # a<b
        print("a<b")
    }
    \end{minted}
\end{frame}
%%%%%%%%%%%%%%%%%%%%%%%%%%%%%%%%%%%%%%%%%%%%%%%%%%%%%%%
\begin{frame}[fragile]\frametitle{Sintaxis IV: while}
    \begin{minted}{R}
  i <- 0
  while (i < 6) {
      i <- i + 1
      if (i == 3) {
        next
      }
  print(i)
  } 
    \end{minted}
\end{frame}
%%%%%%%%%%%%%%%%%%%%%%%%%%%%%%%%%%%%%%%%%%%%%%%%%%%%%%%
\begin{frame}[fragile]\frametitle{Sintaxis V: for}
    \begin{minted}{R}
  for (x in 1:10) {
    print(x)
  } 
    \end{minted}
    \begin{minted}{R}
  fruits <- list("apple", "banana", "cherry")

  for (x in fruits) {
    print(x)
  } 
    \end{minted}
    \begin{minted}{R}
    \end{minted}
\end{frame}
%%%%%%%%%%%%%%%%%%%%%%%%%%%%%%%%%%%%%%%%%%%%%%%%%%%%%%%
\begin{frame}[fragile]\frametitle{Sintaxis VI: functions}
    \begin{minted}{R}
  my_function <- function(x, y=2) {
    return (5 * x * y)
  }

  print(my_function(3)) # y=2 por defecto
    \end{minted}
\end{frame}
%%%%%%%%%%%%%%%%%%%%%%%%%%%%%%%%%%%%%%%%%%%%%%%%%%%%%%%
\begin{frame}[fragile]\frametitle{Sintaxis VII: Estructuras de datos}
    \begin{itemize}
        \item Vector de un solo tipo: \code{c(1,2,3)},\code{1:10}
        \item Lista: \code{list(1, 2i, TRUE)}
        \item Largo: \code{length(lista)}
        \item Agregar: \code{append(lista, "hola")}
        \item Quitar: \code{lista[-1], lista[-c(1,3)]}
        \item Acceso por rango: \code{lista[2:5]}
    \end{itemize}

*OJO: Listas empiezan en 1.
\end{frame}
%%%%%%%%%%%%%%%%%%%%%%%%%%%%%%%%%%%%%%%%%%%%%%%%%%%%%%%
\begin{frame}[fragile]\frametitle{Sintaxis VIII: Data Frame}
    \begin{minted}{R}
    # Create a data frame
    Data_Frame <- data.frame (
      Training = c("Strength", "Stamina", "Other"),
      Pulse = c(100, 150, 120),
    )
    Data_Frame 

    # Todos equivalentes: 
    Data_Frame[1]
    Data_Frame[["Training"]]
    Data_Frame$Training 
    \end{minted}
\end{frame}
%%%%%%%%%%%%%%%%%%%%%%%%%%%%%%%%%%%%%%%%%%%%%%%%%%%%%%%
\begin{frame}[fragile]\frametitle{Exploring data}
    \begin{minted}{R}
Data.Cars <- mtcars 
dim(Data.Cars)
names(Data.Cars) 
rownames(Data.Cars)
Data.Cars$cyl  # Comparar con Data.Cars["cyl"]
summary(Data.Cars)
max(iris["Petal.Length"]) 
which.max(iris$Petal.Length) # con [] no funciona
mean(iris$Petal.Length)
    \end{minted}
\end{frame}
%%%%%%%%%%%%%%%%%%%%%%%%%%%%%%%%%%%%%%%%%%%%%%%%%%%%%%%
\begin{frame}[fragile]\frametitle{Indexaciones y cálculos}
Si 'which' solo opera sobre vectores, cómo extraigo por nombre? \pause

    \begin{minted}{R}  
    datos = iris$"Petal.Length"
    filas = rownames(iris)
    indx_max = which.max(datos)
    filas[indx_max]
    \end{minted}
\end{frame}
%%%%%%%%%%%%%%%%%%%%%%%%%%%%%%%%%%%%%%%%%%%%%%%%%%%%%%%
\begin{frame}[fragile]\frametitle{Estadística}
    \begin{small}
\begin{minted}{R}
  model = lm(Petal.Length ~ Sepal.Length, data = iris)
  summary(model)
  b = model$coefficients[1]
  m = model$coefficients[2]
  minx = min(iris$Sepal.Length)
  maxx = max(iris$Sepal.Length)
  plot(c(minx, maxx), c(b+minx*m, b+maxx*m), 
      type="l", col="red") # (l)ine
  lines(iris$Sepal.Length, iris$Petal.Length, 
        type="p", col="blue") # (p)oint
\end{minted}
    \end{small}
    \idea{Consola...}
\end{frame}

%%%%%%%%%%%%%%%%%%%%%%%%%%%%%%%%%%%%%%%%%%%%%%%%%%%%%%%
\begin{frame}
    \maketitle
\end{frame}
%%%%%%%%%%%%%%%%%%%%%%%%%%%%%%%%%%%%%%%%%%%%%%%%%%%%%%%
\begin{frame}[noframenumbering]\frametitle{Mini ejercicios}
    \begin{enumerate}
        \item Instalar R en su computador 
        \item Probar la consola interactiva
        \item Ejecutar un script con \code{Rscript}
        \item Hacer los siguientes ejercicios: 
            \begin{center}
                https://www.w3schools.com/r/r\_exercises.asp
            \end{center}
        \item Explorar bases de datos de R
        \item En una base de datos elegida, elegir variables para hacer modelos lineales
        \item Qué puede inferir de \code{summary(modelo)}?
        \item Se pueden agregar funciones no lineales de parámetros:
            \begin{minted}{R}
    lm(y ~ x + I(x^3) + log(x), data=datos)
            \end{minted}

    \end{enumerate}
\end{frame}
%%%%%%%%%%%%%%%%%%%%%%%%%%%%%%%%%%%%%%%%%%%%%%%%%%%%%%%
\end{document}
