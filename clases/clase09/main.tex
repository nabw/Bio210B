\documentclass[14pt,aspectratio=169,xcolor=dvipsnames]{beamer}
\usetheme{SimplePlus}
\usepackage{booktabs}
\usepackage{minted}

\title[short title]{Clase 9: Bash como orquestador}
\subtitle{}
\author[NA Barnafi] {Nicolás Alejandro Barnafi Wittwer}
\institute[UC|CMM] 
{
    Pontificia Universidad Católica de Chile \\
    Centro de Modelamiento Matemático
}

\titlegraphic{
    \vspace{-2.4cm}
    \begin{flushright}
      \includegraphics[height=2.5cm]{../images/logos/puc.png} 
    \end{flushright}
}
\date{}

\begin{document}
%%%%%%%%%%%%%%%%%%%%%%%%%%%%%%%%%%%%%%%%%%%%%%%%%%%%%%%
\begin{frame}
    \maketitle
\end{frame}
%%%%%%%%%%%%%%%%%%%%%%%%%%%%%%%%%%%%%%%%%%%%%%%%%%%%%%%
\begin{frame}\frametitle{Motivación}
    \begin{itemize}
        \item Distintos lenguajes tienen distintas fortalezas (Python, R)
        \item Podemos usar Bash para conectar resultados
        \item Existen software especializado: NextFlow, Snakemake
        \item Bash también es un lenguaje de programación
    \end{itemize}
\end{frame}
%%%%%%%%%%%%%%%%%%%%%%%%%%%%%%%%%%%%%%%%%%%%%%%%%%%%%%%
\begin{frame}\frametitle{Navegación efectiva}
    \begin{itemize}
        \item \code{cd}, \code{ls}, etc
        \item \textbf{Usar Tab}
    \end{itemize}

    \idea{Consola...}
\end{frame}
%%%%%%%%%%%%%%%%%%%%%%%%%%%%%%%%%%%%%%%%%%%%%%%%%%%%%%%
\begin{frame}[fragile]\frametitle{Variables}
    \begin{itemize}
        \item Tienen validez solo en terminal actual
        \item Se pueden mantener en sub-terminales con \code{export}
        \item Sintaxis sin espacios
    \end{itemize}
    
    \begin{minted}{bash}
  $ A=2  # Desaparece luego de 'exit' o 'bash'
  $ export A=2 # Se mantiene dentro de nuevos 'bash'
    \end{minted}
\end{frame}
%%%%%%%%%%%%%%%%%%%%%%%%%%%%%%%%%%%%%%%%%%%%%%%%%%%%%%%
\begin{frame}[fragile]\frametitle{\code{if}}
    \begin{small}
    \begin{columns}
        \begin{column}{0.45\textwidth}
            \begin{minted}{bash}
    if [ condition ]; then {
        # código por ejecutar
    }
    elif [ condition2 ]; then {
        # más código
    }
    else {
        # código final
    }
    fi
            \end{minted}
        \end{column}
        \begin{column}{0.45\textwidth}
            Acá, tenemos notación de operaciones booleanas:
            \begin{itemize}
                \item $>,\geq,<,\leq$: \code{-gt},\code{-ge},\code{-lt},\code{-le}
                \item \code{not} (\code{!}), \code{and} (\code{-a}), \code{or} (\code{-o})
                \item Ver si existe archivo/carpeta/variable: \code{-e}/\code{-d}/\code{-n}\footnote{\code{-n} ve si variable no es string vacío}
            \end{itemize}
        \end{column}
    \end{columns}
    \end{small}
\end{frame}
%%%%%%%%%%%%%%%%%%%%%%%%%%%%%%%%%%%%%%%%%%%%%%%%%%%%%%%
\begin{frame}[fragile]\frametitle{Ejemplo}
    \begin{footnotesize}
    \begin{columns}
        \begin{column}{0.7\textwidth}
            \begin{minted}{bash}
A=2
if [ ${A} -gt 1 ]; then {
    echo GT
}
else {
    echo LT
}
fi
A=""
if [ -n "${A}" ]; then { # Comillas!
    echo "SI"
}
else {
    echo NO
}
fi

            \end{minted}
        \end{column}
        \begin{column}{0.25\textwidth}
            \idea{Consola}
        \end{column}
    \end{columns}
    \end{footnotesize}
\end{frame}
%%%%%%%%%%%%%%%%%%%%%%%%%%%%%%%%%%%%%%%%%%%%%%%%%%%%%%%
\begin{frame}[fragile]\frametitle{\code{for}}
    \begin{minted}{bash}
    for n in a b c; do
       echo "$n"
    done
    \end{minted}
    Wildcards:
    \begin{itemize}
        \item \code{*.csv} $\to$ todos los archivos terminados en csv
        \item \code{*.\{csv,xls,txt\}} $\to$ todos los archivos csv, xls, o txt
    \end{itemize}

    \idea{Consola}
\end{frame}
%%%%%%%%%%%%%%%%%%%%%%%%%%%%%%%%%%%%%%%%%%%%%%%%%%%%%%%
\begin{frame}[fragile]\frametitle{Funciones}
 % Falta scripts, source, bash y los 'ifs'
    \begin{minted}{bash}
function helloWorld {
  echo "Hello World"
}
# Equivalente: 
helloWorld () {
    echo "Hello World"
}
    \end{minted}
\end{frame}
%%%%%%%%%%%%%%%%%%%%%%%%%%%%%%%%%%%%%%%%%%%%%%%%%%%%%%%
\begin{frame}[fragile]\frametitle{Funciones: argumentos}
 % Falta scripts, source, bash y los 'ifs'
    \begin{minted}{bash}
function helloWorld {
  echo "$1" # o echo "${1}"
}
    \end{minted}
    \begin{itemize}
        \item \code{\$1}, \code{\$2}, ....: parámetros en orden
        \item \code{\$0}: nombre de función
        \item \code{\$\#}: Número de argumentos   
        \item \code{\$@}: Todos los argumentos
    \end{itemize}
\idea{Consola}
\end{frame}

%%%%%%%%%%%%%%%%%%%%%%%%%%%%%%%%%%%%%%%%%%%%%%%%%%%%%%%
\begin{frame}[fragile]\frametitle{source vs bash}
    \begin{itemize}
        \item Source: Ejecutar en proceso actual
        \begin{minted}{bash}
            $ source script.sh
        \end{minted}
        \item Bash: Crear sub-proceso
        \begin{minted}{bash}
            $ sh script.sh # O 'bash script.sh'
        \end{minted}
    \end{itemize}

Consecuencias: 
    \begin{enumerate}
        \item se pueden perder variables (\code{export})
        \item  si source sale (\code{exit}), se cierra consola
    \end{enumerate}

\idea{Consola}

\end{frame}
%%%%%%%%%%%%%%%%%%%%%%%%%%%%%%%%%%%%%%%%%%%%%%%%%%%%%%%
\begin{frame}[fragile]{Pipes y redirecciones}
    \begin{footnotesize}
    \begin{minted}{bash}
  $ echo "hola"  # Imprime 'hola'
  $ # Toma el texto generado por 'echo' y lo pone en 'out.txt'
  $ echo "hola" > out.txt 
  $ # Toma el texto y lo pone al final de 'out.txt'
  $ echo "hola" >> out.txt 
  $ # Toma salida de 'cat out.txt' y lo entrega como input a grep
  $ cat 'out.txt' | grep "ho"
  $ # Tome la salida del primer comando, 1) la imprime y 2) la guarda en out.txt
  $ cat 'in.txt' | tee out.txt  # -a para 'append'
  $ # Entrega contenido de 'in.txt' a función 'echo'
  $ echo < in.txt
    \end{minted} 
    \end{footnotesize}

\idea{Consola}
\end{frame}
%%%%%%%%%%%%%%%%%%%%%%%%%%%%%%%%%%%%%%%%%%%%%%%%%%%%%%%
\begin{frame}[fragile]\frametitle{Orquestación: Ejercicio}
    Creemos: 
    \begin{enumerate}
        \item Un programa en Python que calcula el promedio de números dados
        \item Llamemos este programa desde un script en Bash
        \item Hagamos que el script en Bash abra un archivo de números y los de a Python
        \item Mostrar resultado final
    \end{enumerate}
\end{frame}
%%%%%%%%%%%%%%%%%%%%%%%%%%%%%%%%%%%%%%%%%%%%%%%%%%%%%%%
\begin{frame}\frametitle{Recap}
    \begin{itemize}
        \item En Python/R es fácil crear funciones específicas
        \item Estas funciones se pueden orquestar desde Bash
        \item Bash tiene muchas funcionalidades para manipular texto
        \item ... esto es solo la punta del iceberg
    \end{itemize}
\end{frame}

%%%%%%%%%%%%%%%%%%%%%%%%%%%%%%%%%%%%%%%%%%%%%%%%%%%%%%%
\begin{frame}
    \maketitle
\end{frame}
%%%%%%%%%%%%%%%%%%%%%%%%%%%%%%%%%%%%%%%%%%%%%%%%%%%%%%%
\begin{frame}[noframenumbering]\frametitle{Mini ejercicios}
    \begin{itemize}
        \item Crear script en bash que lee la variable de ambiente 'A' y dice si es que la variable es mayor a 2.
        \item Crear script en bash que muestre el nombre y contenido de todos los archivos \code{.txt} en la carpeta.
        \item Un programa en Python que calcula el promedio de números dados
        \item Llamemos este programa desde un script en Bash
        \item Hagamos que el script en Bash abra un archivo de números y los de a Python
        \item Mostrar resultado final
    \end{itemize}
\end{frame}
%%%%%%%%%%%%%%%%%%%%%%%%%%%%%%%%%%%%%%%%%%%%%%%%%%%%%%%
\end{document}
