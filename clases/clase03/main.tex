\documentclass[14pt,aspectratio=169,xcolor=dvipsnames]{beamer}
\usetheme{SimplePlus}
\usepackage{booktabs}
\usepackage{minted}
\usepackage{fancyvrb}

\title[short title]{Clase 3: Estructuras de datos e iteradores}
\subtitle{}
\author[NA Barnafi] {Nicolás Alejandro Barnafi Wittwer}
\institute[UC|CMM] 
{
    Pontificia Universidad Católica de Chile \\
    Centro de Modelamiento Matemático
}

\titlegraphic{
    \vspace{-1.8cm}
    \begin{flushright}
      \includegraphics[height=2.5cm]{../images/logos/puc.png} 
    \end{flushright}
}
\date{}


\begin{document}
%%%%%%%%%%%%%%%%%%%%%%%%%%%%%%%%%%%%%%%%%%%%%%%%%%%%%%%
\begin{frame}
    \maketitle
\end{frame}
%%%%%%%%%%%%%%%%%%%%%%%%%%%%%%%%%%%%%%%%%%%%%%%%%%%%%%%
%%%%%%%%%%%%%%%%%%%%%%%%%%%%%%%%%%%%%%%%%%%%%%%%%%%%%%%
\section{Estructuras de datos}
%%%%%%%%%%%%%%%%%%%%%%%%%%%%%%%%%%%%%%%%%%%%%%%%%%%%%%%
%%%%%%%%%%%%%%%%%%%%%%%%%%%%%%%%%%%%%%%%%%%%%%%%%%%%%%%
\begin{frame}[fragile]\frametitle{Motivación}
En el ejemplo del IMC, se vuelve impracticable tener muchos datos:
    \begin{minted}{python}
        peso1 = 1.
        altura1 = 1.0
        peso2 = 2.0 
        altura2 = 2.0
        ...
        peso1200 = 3.0
        altura1200 = 3.0
    \end{minted}

Nos gustaría tener una manera de manejar cantidades \emph{arbitrarias} de datos y operar sobre ellas.
\end{frame}
%%%%%%%%%%%%%%%%%%%%%%%%%%%%%%%%%%%%%%%%%%%%%%%%%%%%%%%
\begin{frame}[fragile]\frametitle{Sintaxis de loop de iteración}
    \begin{minted}{python}
      lista = [1,2,3,4,5]
      for e in lista:
        print(e)
    \end{minted}
\pause El resultado será

1\\
2\\
3\\
4\\
5
\end{frame}
%%%%%%%%%%%%%%%%%%%%%%%%%%%%%%%%%%%%%%%%%%%%%%%%%%%%%%%
\begin{frame}\frametitle{Un poco de sintaxis}
    $$ \texttt{a = [1,2,3,4,5,8]} $$
    \begin{itemize}
        \item<+-> 'a' es una \emph{lista}
        \item<+-> 'a' es un \emph{iterable} (se puede usar en ciclo \texttt{for})
        \item<+-> 'a' tiene componentes (\texttt{a[0],a[1],...})
        \item<+-> Listas no tienen tipos específicos: 
            $$ \texttt{a = [1,'hola',-7.2]} $$
        \item<+-> \texttt{range(10)} para iterar sobre números
    \end{itemize}
\end{frame}
%%%%%%%%%%%%%%%%%%%%%%%%%%%%%%%%%%%%%%%%%%%%%%%%%%%%%%%
\begin{frame}\frametitle{Ejemplo: promedio}
\idea{Consola para ver sus funciones... (\texttt{dir})}
\end{frame}
%%%%%%%%%%%%%%%%%%%%%%%%%%%%%%%%%%%%%%%%%%%%%%%%%%%%%%%
\begin{frame}\frametitle{Otros tipos}
    \begin{itemize}
        \item Tuplas o listas inmodificables: 
            $$  \texttt{a = (1,2,7,8)} $$
        \item Diccionarios o listas indexadas:
            $$ \texttt{a = \{'altura': 1.80, 'peso': 90\}} $$
            *Acceso con \texttt{a['altura']}
        \item Conjuntos o listas no-ordenadas: 
            $$ \texttt{a = \{1,5,9\}} $$
    \end{itemize}

\pause\idea{Exploremos las funciones de cada uno (\texttt{dir})}
\end{frame}
%%%%%%%%%%%%%%%%%%%%%%%%%%%%%%%%%%%%%%%%%%%%%%%%%%%%%%%
\begin{frame}[fragile]\frametitle{Funciones mágicas}

\begin{itemize}
    \item Todas las funciones: $\texttt{dir(a)}$.
    \item Funciones mágicas: \verb+__txt__+
    \item Tipos sumables: %\texttt{\_\_add\_\_}
        \begin{verbatim}
    int.__add__(2,3)
        \end{verbatim}
    \item Subíndices: \verb+__getitem__+
        \begin{verbatim}
    a = [6,7,8]
    list.__getitem__(a,1) # a[1]
        \end{verbatim}
\end{itemize}

\vspace{2cm}
\idea{Veamos...}
\end{frame}
%%%%%%%%%%%%%%%%%%%%%%%%%%%%%%%%%%%%%%%%%%%%%%%%%%%%%%%
%%%%%%%%%%%%%%%%%%%%%%%%%%%%%%%%%%%%%%%%%%%%%%%%%%%%%%%
\section{Iteraciones}
%%%%%%%%%%%%%%%%%%%%%%%%%%%%%%%%%%%%%%%%%%%%%%%%%%%%%%%
%%%%%%%%%%%%%%%%%%%%%%%%%%%%%%%%%%%%%%%%%%%%%%%%%%%%%%%
\begin{frame}[fragile]\frametitle{Motivación}
    Todo bien con el ciclo for, pero...

    \begin{itemize}
        \item A veces no sabemos cuántas veces iterar
        \item A veces iteramos hasta cumplir algún criterio. Ej: estudiar
    \end{itemize}
\end{frame}
%%%%%%%%%%%%%%%%%%%%%%%%%%%%%%%%%%%%%%%%%%%%%%%%%%%%%%%
\begin{frame}[fragile]\frametitle{Sintaxis de loop de iteración}
    \begin{minted}{python}
        while condition: # condition es un bool
            # código
            # se vuelve a ejecutar 
            # mientras 'condition' sea True
    \end{minted}
    
    \vspace{1cm}
    \begin{itemize}
        \item Lavarse las manos
        \item Estudiar
        \item Afilar cuchillo
    \end{itemize}
\end{frame}
%%%%%%%%%%%%%%%%%%%%%%%%%%%%%%%%%%%%%%%%%%%%%%%%%%%%%%%
\begin{frame}[fragile]\frametitle{Ejercicio: ahorros con \texttt{break/continue}}
\begin{small}
    \begin{minted}{python}
        plata = 0
        dias = 0 # contador
        while True: 
            if dia == "sabado" or dia == "domingo": 
                dia = avanzarDia(dia)
                dias += 1
                continue # Volver al inicio del while
            plata += trabajar()
            if plata >= objetivo:
                break # Salir del while
            dia = avanzarDia(dia)
            dias += 1
        print("Meta alcanzada en {} días".format(dias))
    \end{minted}
    
\alertGreen{Sirven en \texttt{for} también!} Extra: replicar con if/elif
\end{small}
\end{frame}
%%%%%%%%%%%%%%%%%%%%%%%%%%%%%%%%%%%%%%%%%%%%%%%%%%%%%%%
\begin{frame}[fragile]\frametitle{Trucos de python}
    \begin{minted}{python}
        apellidos = ["Nguyen", "Janssen", "Smith"]
        ids = [100,101,102]
    \end{minted}
    \begin{itemize}
        \item Juntar listas:
        \begin{minted}{python}
        for apellido,id in zip(apellidos,ids):
            # [...]
    \end{minted}
        \item Agregar contador:
    \begin{minted}{python}
        for i,apellido in enumerate(apellidos):
            # [...]
    \end{minted}
    \end{itemize}
\end{frame}
%%%%%%%%%%%%%%%%%%%%%%%%%%%%%%%%%%%%%%%%%%%%%%%%%%%%%%%
\begin{frame}\frametitle{Practiquemos un poco}
\idea{Consola...}
\end{frame}
%%%%%%%%%%%%%%%%%%%%%%%%%%%%%%%%%%%%%%%%%%%%%%%%%%%%%%%
\begin{frame}[fragile]\frametitle{Mini ejercicios}
    \begin{enumerate}
        \item Dividir un número dado por 2 hasta que sea menor que $10^{-4}$ y reportar número de divisiones
        \item Crear una lista con números hasta 1000 y luego quitar todos los impares   
    \end{enumerate}
    \begin{minted}{python}
        a = 100
        a % 2 # Resto de dividir por 2
    \end{minted}
\end{frame}
%%%%%%%%%%%%%%%%%%%%%%%%%%%%%%%%%%%%%%%%%%%%%%%%%%%%%%%

\begin{frame}
    \maketitle
\end{frame}
\end{document}
