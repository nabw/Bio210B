\documentclass[14pt,aspectratio=169,xcolor=dvipsnames]{beamer}
\usetheme{SimplePlus}
\usepackage{booktabs}
\usepackage{minted}

\title[short title]{Clase 5: Funciones}
\subtitle{}
\author[NA Barnafi] {Nicolás Alejandro Barnafi Wittwer}
\institute[UC|CMM] 
{
    Pontificia Universidad Católica de Chile \\
    Centro de Modelamiento Matemático
}

\titlegraphic{
    \vspace{-1.8cm}
    \begin{flushright}
      \includegraphics[height=2.5cm]{../images/logos/puc.png} 
    \end{flushright}
}

%\date{26/07/2024, Vancouver, WCCM}
\setbeamercovered{transparent}

\begin{document}
%%%%%%%%%%%%%%%%%%%%%%%%%%%%%%%%%%%%%%%%%%%%%%%%%%%%%%%
\begin{frame}
    \maketitle
\end{frame}
%%%%%%%%%%%%%%%%%%%%%%%%%%%%%%%%%%%%%%%%%%%%%%%%%%%%%%%
\begin{frame}[fragile]\frametitle{Motivación}
Supongamos queremos calcular IMC

\rule{\textwidth}{1pt}
\scriptsize
\begin{minted}{python}
    # Enlistar algunos pesos y alturas
    peso1 = 62.2
    altura1 = 1.50
    peso2 = 94.5
    altura2 = 1.87
    
    # Calcular IMC
    imc1 = peso1 / altura1 / altura1 # p / h^2. Por qué?
    imc2 = peso2 / altura2 / altura2
\end{minted}
\rule{\textwidth}{1pt}
\pause Sería mejor... 
\begin{minted}{python}
    imc1 = calcularIMC(peso1, altura1)
    imc2 = calcularIMC(peso2, altura2)
\end{minted}
\rule{\textwidth}{1pt}

\end{frame}
%%%%%%%%%%%%%%%%%%%%%%%%%%%%%%%%%%%%%%%%%%%%%%%%%%%%%%%
\begin{frame}\frametitle{Motivación}
    \begin{itemize}
        \item Las funciones \emph{reducen complejidad}
        \item Las funciones \emph{vuelven código reutilizable}
    \end{itemize}
\end{frame}
%%%%%%%%%%%%%%%%%%%%%%%%%%%%%%%%%%%%%%%%%%%%%%%%%%%%%%%
\begin{frame}[fragile]\frametitle{Sintaxis}

\begin{minted}{python}
    def functionName(arg1, arg2, ...):
        # [...]
        return expression # Retorno tiene tipo 
        # Puede no estar (return None)
\end{minted}
\rule{\textwidth}{1pt}
Por ejemplo, en el caso del IMC
\begin{minted}{python}
    def calcularIMC(peso, altura):
        # [...]
        return peso / altura**2
\end{minted}
\rule{\textwidth}{1pt}
\idea{Ojo con conflictos de nombre (dentro y fuera de función)}
%\idea{Ojo: Tipos de retorno}
\end{frame}
%%%%%%%%%%%%%%%%%%%%%%%%%%%%%%%%%%%%%%%%%%%%%%%%%%%%%%%
\begin{frame}[fragile]\frametitle{La función \texttt{print}}
Esta función muestra el valor de una variable. 
\begin{itemize}
    \item Archivo de ejemplo 'test.py':
    \begin{minted}{python}
    a = 2
    b = 3*a
    print(b)
    \end{minted}
    \item Lo que vemos al ejecutarlo: 

        \begin{tabular}{l}
            \texttt{ \$ python test.py} \\
            \texttt{ \$ 6}
        \end{tabular}
\end{itemize}

\idea{Veamos algunos ejemplos en consola}
\end{frame}
%%%%%%%%%%%%%%%%%%%%%%%%%%%%%%%%%%%%%%%%%%%%%%%%%%%%%%%
\begin{frame}[fragile]\frametitle{Argumentos por keyword}
    Podemos definir argumentos \emph{por defecto}
    \begin{minted}{python}
    def printName(nombre="Nicolás", apellido="Barnafi")
        print("Me llamo ", nombre, " ", apellido)

    print("nicolás", "barnafi") # en orden
    print("nicolás")            # 1 por defecto
    print()                     # 2 por defecto
    print(nombre="nico")        # keyword arg
    print(apellido="bfi")       # keyword arg
    \end{minted}
\end{frame}
%%%%%%%%%%%%%%%%%%%%%%%%%%%%%%%%%%%%%%%%%%%%%%%%%%%%%%%
\begin{frame}[fragile]\frametitle{Sintaxis que no encontrarán detallada}
    \begin{minted}{python}
    def printName(nombre="Nicolás", apellido="Barnafi")
        print("Me llamo ", nombre, " ", apellido)

    lista = ["nico", "barnafi"]
    diccionario = {"nombre": "nico", 
                   "apellido": "barnafi"}
    print(*lista) # por lista
    print(**kwargs) # por diccionario
    \end{minted}

\end{frame}
%%%%%%%%%%%%%%%%%%%%%%%%%%%%%%%%%%%%%%%%%%%%%%%%%%%%%%%
\begin{frame}\frametitle{Ejemplo}
\idea{Veamos cómo calcular el área de círculos}

$$ A(r) = \pi r^2 $$
\idea{Qué pasa con los argumentos equivocados?}

\vspace{1cm}
\pause Type-checking: \texttt{type(a) == int}
\end{frame}
%%%%%%%%%%%%%%%%%%%%%%%%%%%%%%%%%%%%%%%%%%%%%%%%%%%%%%%
\begin{frame}[fragile]\frametitle{Scope o alcance de nombres}
Qué pasa en este código?

\begin{minted}{python}
a = 2
def f(a):
    b = a**2
    return b
print(f(3))
\end{minted}

\end{frame}
%%%%%%%%%%%%%%%%%%%%%%%%%%%%%%%%%%%%%%%%%%%%%%%%%%%%%%%
\begin{frame}\frametitle{Recap}
    \begin{itemize}
        \item Qué rol cumplen las funciones
        \item Cómo definir funciones
        \item Vimos el alcance (scope) de variables
        \item Muchos ejemplos :)
    \end{itemize}
\end{frame}
%%%%%%%%%%%%%%%%%%%%%%%%%%%%%%%%%%%%%%%%%%%%%%%%%%%%%%%
\begin{frame}
    \maketitle
\end{frame}
\end{document}
