\documentclass[14pt,aspectratio=169,xcolor=dvipsnames]{beamer}
\usetheme{SimplePlus}
\usepackage{minted}
\usepackage{booktabs}

\title[short title]{Clase 2: Lenguajes de programación y Python}
\subtitle{}
\author[NA Barnafi] {Nicolás Alejandro Barnafi Wittwer}
\institute[UC|CMM] 
{
    Pontificia Universidad Católica de Chile \\
    Centro de Modelamiento Matemático
}

\titlegraphic{
    \vspace{-1.8cm}
    \begin{flushright}
      \includegraphics[height=2.5cm]{../images/logos/puc.png} 
    \end{flushright}
}

\date{07/08/2024}
\setbeamercovered{transparent}

\begin{document}
%%%%%%%%%%%%%%%%%%%%%%%%%%%%%%%%%%%%%%%%%%%%%%%%%%%%%%%
\begin{frame}
    \maketitle
\end{frame}
%%%%%%%%%%%%%%%%%%%%%%%%%%%%%%%%%%%%%%%%%%%%%%%%%%%%%%%
\section{Compilación de código}
%%%%%%%%%%%%%%%%%%%%%%%%%%%%%%%%%%%%%%%%%%%%%%%%%%%%%%%
\begin{frame}[t]\frametitle{La compilación}
    \idea{Consiste en traducir nuestro lenguaje al del PC}

    \vspace{1cm}
    \begin{itemize}
        \item<+-> Lo que nosotros entendemos: Lenguaje
        \item<+-> Lo que entiende el computador: 
    
        100010101010110101011010110101011011010101011101010000
            
    \end{itemize}
\end{frame}
%%%%%%%%%%%%%%%%%%%%%%%%%%%%%%%%%%%%%%%%%%%%%%%%%%%%%%%
\begin{frame}[fragile]\frametitle{Ejemplo de compilación}
\begin{small}
    \begin{columns}[t]
        \begin{column}[b]{0.4\textwidth}
            Para humanos:
            \vspace{2.0cm}
            \begin{minted}{C}
#include <stdio.h>
   int main() {
   printf("Hello, World!\n");
   return 0;
            }
            \end{minted}

    \vfill
        \end{column}
        \begin{column}{0.01\textwidth}
            \rule{0.4pt}{6cm}

        \end{column}
        \begin{column}[b]{0.45\textwidth}
            Para PC: 
            \begin{minted}{gas}
    .file    "hello.c"
    .text
    .section .rodata
.LC0:
    .string  "Hello, World!"
    .text
    .globl   main
    .type    main, @function
main:
            \end{minted}

    \vfill
        \end{column}
    \end{columns}
\end{small}
\end{frame}
%%%%%%%%%%%%%%%%%%%%%%%%%%%%%%%%%%%%%%%%%%%%%%%%%%%%%%%
\begin{frame}\frametitle{El trabajo de compilación}
    \begin{enumerate}
        \item Procesar el archivo con código
        \item Generar símbolos de programas
        \item Juntar símbolos para generar programa
    \end{enumerate}

    \pause \begin{center}
        \begin{tabular}{c | c}
           Pros    &   Contras \\ \midrule
         \alertGreen{Eficiente} &   \alert{Inflexible}  \\
        \end{tabular}
    \end{center}

\pause Ejemplos de programas compilados: \texttt{ls, cd, }$\hdots$
\end{frame}
%%%%%%%%%%%%%%%%%%%%%%%%%%%%%%%%%%%%%%%%%%%%%%%%%%%%%%%
\begin{frame}\frametitle{Alternativa}
Lenguajes compilados son eficientes pero...
    \begin{itemize}
        \item Más complejos
        \item Requieren recompilar constantemente
    \end{itemize}

\vspace{1cm}
\pause Alternativa: \textbf{Lenguajes interpretados}

\pause
    \begin{itemize}
        \item Muchos subprogramas pre-compilados
        \item No se compila, se ejecuta secuencialmente (una línea a la vez)
        \item Permiten una consola interactiva
    \end{itemize}
\end{frame}
%%%%%%%%%%%%%%%%%%%%%%%%%%%%%%%%%%%%%%%%%%%%%%%%%%%%%%%
\begin{frame}\frametitle{Algunos ejemplos}
    \begin{columns}
        \begin{column}{0.45\textwidth}
            Compilados
            \begin{itemize}
                \item C/C++/C\#
                \item Java
                \item Rust
            \end{itemize}
        \end{column}

        \begin{column}{0.45\textwidth}
            Interpretados
            \begin{itemize}
                \item Python
                \item R
                \item MATLAB
                \item Julia
            \end{itemize}
        \end{column}
    \end{columns}
\end{frame}
%%%%%%%%%%%%%%%%%%%%%%%%%%%%%%%%%%%%%%%%%%%%%%%%%%%%%%%
\begin{frame}\frametitle{Ranking de lenguajes}
    \begin{center}
        \includegraphics[width=0.4\textwidth]{../images/top-lenguajes.png}
    \end{center}
\end{frame}
%%%%%%%%%%%%%%%%%%%%%%%%%%%%%%%%%%%%%%%%%%%%%%%%%%%%%%%
\begin{frame}\frametitle{Más clasificaciones}
Los lenguajes también puede ser \emph{orientados a objetos}. Esto determina el nivel posible de abstracción.
    \begin{columns}
        \begin{column}{0.45\textwidth}
            Con objetos
            \begin{itemize}
                \item C++/C\#
                \item Java
                \item Rust
                \item Python
            \end{itemize}
        \end{column}

        \begin{column}{0.45\textwidth}
            Sin objetos
            \begin{itemize}
                \item C
                \item R
                \item MATLAB
                \item Julia
            \end{itemize}
        \end{column}
    \end{columns}
\end{frame}
%%%%%%%%%%%%%%%%%%%%%%%%%%%%%%%%%%%%%%%%%%%%%%%%%%%%%%%
\section{Python}
%%%%%%%%%%%%%%%%%%%%%%%%%%%%%%%%%%%%%%%%%%%%%%%%%%%%%%%
\begin{frame}\frametitle{Datos sobre Python}
    \begin{itemize}
        \item Lenguaje interpretado, orientado a objetos
        \item Creado en 1991 por Guido van Rossum
        \item Basado en C
        \item Librerías de IA: PyTorch, Keras, TensorFlow
        \item MUCHAS librerías científicas
        \item Ejecuta extensión \texttt{'.py'}
    \end{itemize}

\begin{block}{}
    \emph{[...] as a hobby project to keep himself busy during the Christmas holidays of 1989. }
\end{block}
\end{frame}
%%%%%%%%%%%%%%%%%%%%%%%%%%%%%%%%%%%%%%%%%%%%%%%%%%%%%%%
\begin{frame}\frametitle{Cómo usar Python}
    \begin{enumerate}
        \item<1-> Como programa
            \only<1>{
                \begin{flushright}
                    \texttt{nico@nico-pc:\~/Codes\$ python test.py}
                \end{flushright}
            }
        \item<2-> Como consola interactiva
            \only<2>{
                \begin{flushright}
                    \texttt{nico@nico-pc:\~/Codes\$ python} 
                    \includegraphics[width=0.9\textwidth]{../images/python-intepreter.png}
                \end{flushright}
            }
        \item<3-> \textbf{Jupyter/Collab/IDE (ayudantía)}
    \end{enumerate}

\vspace{1cm}
\pause\idea{Veamos un poco la consola}
\end{frame}
%%%%%%%%%%%%%%%%%%%%%%%%%%%%%%%%%%%%%%%%%%%%%%%%%%%%%%%
\begin{frame}[t]\frametitle{Primeros códigos}
    \begin{itemize}
        \item Definición de variables (case sensitive)
            $$ > \underbrace{\texttt{a = 2}}_\text{Asignación \emph{siempre} de derecha a izquierda} $$
        \item Toda variable tiene un tipo 
            \begin{itemize}
                \item \texttt{> a = 2} : \hspace{0.3cm}\>\>\>\>\> 'a' es un 'int' (entero)
                \item \texttt{> a = 3.14}: \>\>\>\>\>'a' es un 'float' (punto flotante)
                \item \texttt{> a = "hola"}: 'a' es un 'string' (texto)
                \item \texttt{> a = True}: \>\>\> 'a' es un 'bool' (booleano)\footnote{esto tendrá sentido más adelante}
            \end{itemize}
        \item Se pueden dejar comentarios con \#
    \end{itemize}

\idea{Veamos en consola usando la función \texttt{type}}
\end{frame}
%%%%%%%%%%%%%%%%%%%%%%%%%%%%%%%%%%%%%%%%%%%%%%%%%%%%%%%
\begin{frame}\frametitle{Aritmética y propiedades algebráicas}
Cuál es el valor resultante de las siguientes operaciones? 

\vspace{1cm}
    \begin{itemize}
        \item \texttt{> a = 2+2}
        \item \texttt{> a = 2+2*2}
        \item \texttt{> a = 2/2*2}
        \item \texttt{> a = 2**2}   \textcolor{gray}{(elevado)}
        \item \texttt{> a = 'ho' + 'la'}
    \end{itemize}

\idea{Ojo: división entre enteros da un 'float'}

\end{frame}
%%%%%%%%%%%%%%%%%%%%%%%%%%%%%%%%%%%%%%%%%%%%%%%%%%%%%%%
\begin{frame}\frametitle{Nota sobre la división}
La división se entiende como aplicada exclusivamente al número precedente. Eso implica lo siguiente: 
    \begin{itemize}
        \item \texttt{2/2} es $\frac 2 2$
        \item \texttt{2/2*2} es $\left(\frac 2 2\right) \times 2$
        \item \texttt{2/2/2} es $\frac{\left(\frac 2 2\right)}{2}= \frac{2}{2\times 2}$
    \end{itemize}

\pause \idea{Es más fácil imaginar $a/b$ como $a b^{-1}$}
\end{frame}
%%%%%%%%%%%%%%%%%%%%%%%%%%%%%%%%%%%%%%%%%%%%%%%%%%%%%%%
\begin{frame}\frametitle{Álgebra booleana}
    Keywords importantes: 
    \begin{itemize}
        \item \texttt{and}, \texttt{or}, \texttt{not}
        \item \texttt{==}, \texttt{<=}, \texttt{>=}, \texttt{<}, \texttt{>}
    \end{itemize}

En términos de precedencia, \alertBlue{\texttt{and}} es un producto y \alertBlue{\texttt{or}} es una suma: 

$$ \texttt{a and b or c} = (\texttt{a and b}) \texttt{ or } c $$

Notar que el resultado es siempre otro booleano.
\end{frame}
%%%%%%%%%%%%%%%%%%%%%%%%%%%%%%%%%%%%%%%%%%%%%%%%%%%%%%%
\begin{frame}\frametitle{Recap}
    \begin{itemize}
        \item Concepto de compilación
        \item Lenguaje interpretado vs. compilado
        \item Python: lenguaje interpretado, orientado a objetos
        \item Usamos la consola interactiva de Python
        \item Tipos de variables: int, float, bool, string
        \item Operaciones entre variables, aritmética y álgebra

    \end{itemize}
\end{frame}
%%%%%%%%%%%%%%%%%%%%%%%%%%%%%%%%%%%%%%%%%%%%%%%%%%%%%%%
\begin{frame}
    \maketitle
\end{frame}
\end{document}
