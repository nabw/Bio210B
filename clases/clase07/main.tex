\documentclass[14pt,aspectratio=169,xcolor=dvipsnames]{beamer}
\usetheme{SimplePlus}
\usepackage{booktabs}
\usepackage{minted}

\title[short title]{Clase 7: Documentación y colaboración}
\subtitle{}
\author[NA Barnafi] {Nicolás Alejandro Barnafi Wittwer}
\institute[UC|CMM] 
{
    Pontificia Universidad Católica de Chile \\
    Centro de Modelamiento Matemático
}

\titlegraphic{
    \vspace{-1.8cm}
    \begin{flushright}
      \includegraphics[height=2.5cm]{../images/logos/puc.png} 
    \end{flushright}
}
\date{}

\begin{document}
%%%%%%%%%%%%%%%%%%%%%%%%%%%%%%%%%%%%%%%%%%%%%%%%%%%%%%%
\begin{frame}
    \maketitle
\end{frame}
%%%%%%%%%%%%%%%%%%%%%%%%%%%%%%%%%%%%%%%%%%%%%%%%%%%%%%%
%%%%%%%%%%%%%%%%%%%%%%%%%%%%%%%%%%%%%%%%%%%%%%%%%%%%%%%
\section{Documentación}
%%%%%%%%%%%%%%%%%%%%%%%%%%%%%%%%%%%%%%%%%%%%%%%%%%%%%%%
%%%%%%%%%%%%%%%%%%%%%%%%%%%%%%%%%%%%%%%%%%%%%%%%%%%%%%%
\begin{frame}\frametitle{Motivación}
    \begin{itemize}
        \item El código \emph{envejece}
        \item Uno \textbf{no} recuerda por qué hizo cosas
        \item Los futuros usuarios de nuestro código \huge{menos}
    \end{itemize}
\end{frame}
%%%%%%%%%%%%%%%%%%%%%%%%%%%%%%%%%%%%%%%%%%%%%%%%%%%%%%%
\begin{frame}[fragile]\frametitle{Docs de una función}
    \begin{minted}{python}
    def suma(x,y):
        """
        Esta función entrega la suma de sus argumentos. 
        args: 
            x: Un elemento sumable
            y: Un segundo elemento sumable
        Return: suma de entradas (Int)
        """
        z = x+y
        return z
    \end{minted}
\end{frame}
%%%%%%%%%%%%%%%%%%%%%%%%%%%%%%%%%%%%%%%%%%%%%%%%%%%%%%%
\begin{frame}[fragile]\frametitle{Docs de una función}
    \begin{minted}{python}
    >>> help(suma)
    \end{minted}

    \vspace{1cm}
    Se guarda en la variable mágica \code{suma.\_\_doc\_\_}

    \idea{Consola}
\end{frame}

%%%%%%%%%%%%%%%%%%%%%%%%%%%%%%%%%%%%%%%%%%%%%%%%%%%%%%%
\begin{frame}[fragile]\frametitle{Docs de una clase}
    \begin{footnotesize}
    \begin{minted}{python}
    class Vec2D:
        """
        Clase que representa un vector en R2.
        """
        def __init__(self,x,y):
            """
            Constructor, toma como parámetros coordenadas x e y.
            """
            self.x = x
            self.y = y
        def norma(self):
            """
            Entrega la norma euclideana del vector. Requiere 'sqrt'
            """
            return sqrt(self.x**2 + self.y**2)
    \end{minted}

    \vspace{-1cm}
    \hfill    \idea{Consola}
    \end{footnotesize}
\end{frame}
%%%%%%%%%%%%%%%%%%%%%%%%%%%%%%%%%%%%%%%%%%%%%%%%%%%%%%%
\begin{frame}[fragile]\frametitle{Documentación avanzada}
    Existe mucho software (Sphinx, Doxygen, etc). Veremos \code{Mkdocs}.

    \begin{minted}{bash}
    # pip install mkdocs
    # mkdocs new proyecto
    # cd proyecto
    # mkdocs serve
    # mkdocs build
    \end{minted} 

    \vspace{1cm}
    Se usa formato Markdown (popular por ser simple)

\end{frame}
%%%%%%%%%%%%%%%%%%%%%%%%%%%%%%%%%%%%%%%%%%%%%%%%%%%%%%%
\begin{frame}{Mkdocs}
    Veamos su documentación: \\https://www.mkdocs.org/getting-started/
\end{frame}
%%%%%%%%%%%%%%%%%%%%%%%%%%%%%%%%%%%%%%%%%%%%%%%%%%%%%%%
\begin{frame}[fragile]\frametitle{Otra opción}
    \begin{minted}{bash}
    # pip install pdoc3
    # cd proyecto
    # pdoc3 --html .
    \end{minted}

    \idea{Consola}
\end{frame}
%%%%%%%%%%%%%%%%%%%%%%%%%%%%%%%%%%%%%%%%%%%%%%%%%%%%%%%
%%%%%%%%%%%%%%%%%%%%%%%%%%%%%%%%%%%%%%%%%%%%%%%%%%%%%%%
\section{Desarrollo colaborativo}
%%%%%%%%%%%%%%%%%%%%%%%%%%%%%%%%%%%%%%%%%%%%%%%%%%%%%%%
%%%%%%%%%%%%%%%%%%%%%%%%%%%%%%%%%%%%%%%%%%%%%%%%%%%%%%%
\begin{frame}\frametitle{Motivación}
    {\huge
    \begin{center}
        \alertBlue{Cómo compartir código?}
    \end{center}
    }
    \begin{flushright}
    \includegraphics[height=2cm]{../images/logos/git.png}
    \end{flushright}
\end{frame}
%%%%%%%%%%%%%%%%%%%%%%%%%%%%%%%%%%%%%%%%%%%%%%%%%%%%%%%
\begin{frame}[t,fragile]\frametitle{Cómo funciona git}
  \begin{columns}
    \begin{column}[t]{0.45\textwidth}
        Original
        \begin{minted}{python}
def suma(x,y):
    return x+y
        \end{minted}
    \end{column}
    \begin{column}[t]{0.45\textwidth}
        Modificado
        \begin{minted}{python}
def suma(x,y):
    """
    Función que suma
    """
    return x+y
        \end{minted}
    \end{column}
  \end{columns}

  \vspace{1cm}
  \begin{itemize}
  \item Diferencia: 3 líneas agregadas en la línea 2. 

  \item Se guardan solo los \alert{incrementos} de archivos
  \end{itemize}
\end{frame}
%%%%%%%%%%%%%%%%%%%%%%%%%%%%%%%%%%%%%%%%%%%%%%%%%%%%%%%
\begin{frame}\frametitle{Función general (offline)}
    Flujo de trabajo:
    \begin{itemize}
        \item Cambios propuestos de archivos \phantom{ }
            
            \begin{minipage}{0.35\textwidth}
            \begin{block}{}
                Stage \hfill(\code{git add})
            \end{block}
            \end{minipage}

        \item Cambios incorporados \phantom{ }

            \begin{minipage}{0.35\textwidth}
            \begin{block}{}
                Commit \hfill(\code{git commit})
            \end{block}
            \end{minipage}

        \item Deshacer cambios propuestos \phantom{ }
            
            \begin{minipage}{0.35\textwidth}
            \begin{block}{}
                Stage \hfill(\code{git reset})
            \end{block}
            \end{minipage}

    \end{itemize}
\end{frame}
%%%%%%%%%%%%%%%%%%%%%%%%%%%%%%%%%%%%%%%%%%%%%%%%%%%%%%%
\begin{frame}[t,fragile]\frametitle{Ejemplo}
  \begin{columns}
    \begin{column}[t]{0.45\textwidth}
        Original
        \begin{minted}{python}
def suma(x,y):
    return x+y
        \end{minted}
    \end{column}
    \begin{column}[t]{0.45\textwidth}
        Modificado
        \begin{minted}{python}
def suma(x,y):
    """
    Función que suma
    """
    return x+y
        \end{minted}
    \end{column}
  \end{columns}
    
\begin{enumerate}
    \item Tengo archivo original $\to$ lo modifico
    \item Agrego cambios como nueva modificación propuesta \hfill (\alertGreen{stage})
    \item Incorporo modificación \hfill (\alertGreen{commit})
\end{enumerate}
\idea{Consola}
\end{frame}
%%%%%%%%%%%%%%%%%%%%%%%%%%%%%%%%%%%%%%%%%%%%%%%%%%%%%%%%
\begin{frame}\frametitle{Función general (online)}
    Flujo de trabajo:
    \begin{itemize}
        \item Crear carpeta con \emph{copia local} de repositorio online
            \begin{minipage}{0.35\textwidth}
            \begin{block}{}
                Clone \hfill(\code{git clone})
            \end{block}
            \end{minipage}

        \item Descargar contenidos de repositorio online \phantom{ }
            
            \begin{minipage}{0.35\textwidth}
            \begin{block}{}
                Pull \hfill(\code{git pull})
            \end{block}
            \end{minipage}
        \item Enviar cambios creados \phantom{ }

            \begin{minipage}{0.35\textwidth}
            \begin{block}{}
                Commit \hfill(\code{git push})
            \end{block}
            \end{minipage}
    \end{itemize}
\end{frame}
%%%%%%%%%%%%%%%%%%%%%%%%%%%%%%%%%%%%%%%%%%%%%%%%%%%%%%%
\begin{frame}\frametitle{Otras funcionalidades}
    \begin{itemize}
        \item Manejo de conflictos 
        \item Crear ramas para desarrollo paralelo (\code{git branch})
        \item Deshacer cambios (\code{git reset})
        \item Ir a versiones anteriores del código (\code{git checkout})
        \item Manejo de protocolos de seguridad (SSH)
        \item ... y \alert{muchas} más
    \end{itemize}
\end{frame}
%%%%%%%%%%%%%%%%%%%%%%%%%%%%%%%%%%%%%%%%%%%%%%%%%%%%%%%
\begin{frame}\frametitle{Recap}
    \begin{itemize}
        \item Documentación es fundamental
        \item Existen muchas herramientas para hacerlo en modo semi-automático
        \item \code{mkdocs} bonito pero trabajoso
        \item \code{pdoc3} feo pero fácil
        \item \code{git} es LA herramienta de desarrollo colaborativo
        \item Se basa en un grafo de estados
        \item Modificaciones locales: \code{add}, \code{commit}, \code{reset}
        \item Modificaciones remotas: \code{clone}, \code{pull}, \code{push}
    \end{itemize}
\end{frame}
%%%%%%%%%%%%%%%%%%%%%%%%%%%%%%%%%%%%%%%%%%%%%%%%%%%%%%%
\begin{frame}
    \maketitle
\end{frame}
%%%%%%%%%%%%%%%%%%%%%%%%%%%%%%%%%%%%%%%%%%%%%%%%%%%%%%%
\begin{frame}[noframenumbering]\frametitle{Mini ejercicios}
    \begin{enumerate}
        \item Instalar git
        \item Crear cuenta en \code{github.com}
        \item Subir llave SSH
        \item Crear repositorio en su cuenta
        \item Clonar el repositorio localmente
        \item Crear una carpeta de nombre 'proyecto'
        \item Dentro esa carpeta crear un archivo con dos funciones
        \item Instalar pdoc3 y correr \code{pdoc3 --html .} desde la carpeta
        \item Ver documentación en archivo html recién creado

        \item Hacer \emph{push} del cambio y verificar el cambio
    \end{enumerate}
    
    \begin{center}
        Turorial online: \code{https://www.w3schools.com/git/}
    \end{center}
\end{frame}
%%%%%%%%%%%%%%%%%%%%%%%%%%%%%%%%%%%%%%%%%%%%%%%%%%%%%%%
\begin{frame}
    \maketitle
\end{frame}
%%%%%%%%%%%%%%%%%%%%%%%%%%%%%%%%%%%%%%%%%%%%%%%%%%%%%%%

\end{document}
