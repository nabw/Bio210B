\documentclass[14pt,aspectratio=169,xcolor=dvipsnames]{beamer}
\usetheme{SimplePlus}
\usepackage{booktabs}
\usepackage{minted}

\title[short title]{Clase 8: Iteradores}
\subtitle{}
\author[NA Barnafi] {Nicolás Alejandro Barnafi Wittwer}
\institute[UC|CMM] 
{
    Pontificia Universidad Católica de Chile \\
    Centro de Modelamiento Matemático
}

\titlegraphic{
    \vspace{-1.8cm}
    \begin{flushright}
      \includegraphics[height=2.5cm]{../images/logos/puc.png} 
    \end{flushright}
}

%\date{26/07/2024, Vancouver, WCCM}
%\setbeamercovered{transparent}

\begin{document}
%%%%%%%%%%%%%%%%%%%%%%%%%%%%%%%%%%%%%%%%%%%%%%%%%%%%%%%
\begin{frame}
    \maketitle
\end{frame}
%%%%%%%%%%%%%%%%%%%%%%%%%%%%%%%%%%%%%%%%%%%%%%%%%%%%%%%
\begin{frame}[fragile]\frametitle{Motivación}
    Todo bien con el ciclo for, pero...

    \begin{itemize}
        \item A veces no sabemos cuántas veces iterar
        \item A veces iteramos hasta cumplir algún criterio. Ej: estudiar
    \end{itemize}
\end{frame}
%%%%%%%%%%%%%%%%%%%%%%%%%%%%%%%%%%%%%%%%%%%%%%%%%%%%%%%
\begin{frame}[fragile]\frametitle{Sintaxis de loop de iteración}
    \begin{minted}{python}
        while condition: # condition es un bool
            # code, change 'condition'
    \end{minted}
    
    \vspace{1cm}
    \begin{itemize}
        \item Lavarse las manos
        \item Estudiar
        \item Afilar cuchillo
    \end{itemize}
\end{frame}
%%%%%%%%%%%%%%%%%%%%%%%%%%%%%%%%%%%%%%%%%%%%%%%%%%%%%%%
\begin{frame}[fragile]\frametitle{Ejercicio: ahorros con \texttt{break/continue}}
\begin{small}
    \begin{minted}{python}
        plata = 0
        dias = 0 # contador
        white True: 
            if dia == "sabado" or dia == "domingo": 
                dia = avanzarDia(dia)
                dias += 1
                continue # Volver al inicio del while
            plata += trabajar()
            if plata >= objetivo:
                break # Salir del while
            dia = avanzarDia(dia)
        print("Meta alcanzada en {} días".format(dias))
    \end{minted}
    
\alertGreen{Sirven en \texttt{for} también!} Extra: replicar con if/elif
\end{small}
\end{frame}
%%%%%%%%%%%%%%%%%%%%%%%%%%%%%%%%%%%%%%%%%%%%%%%%%%%%%%%
\begin{frame}[fragile]\frametitle{Trucos de python}
    \begin{minted}{python}
        apellidos = ["Nguyen", "Janssen", "Smith"]
        ids = [100,101,102]
    \end{minted}
    \begin{itemize}
        \item Juntar listas:
        \begin{minted}{python}
        for apellido,id in zip(apellidos,ids):
            # [...]
    \end{minted}
        \item Agregar contador:
    \begin{minted}{python}
        for i,apellido in enumerate(apellidos):
            # [...]
    \end{minted}
    \end{itemize}
\end{frame}
%%%%%%%%%%%%%%%%%%%%%%%%%%%%%%%%%%%%%%%%%%%%%%%%%%%%%%%
\begin{frame}\frametitle{Practiquemos un poco}
\idea{Consola...}
\end{frame}
%%%%%%%%%%%%%%%%%%%%%%%%%%%%%%%%%%%%%%%%%%%%%%%%%%%%%%%
\begin{frame}
    \maketitle
\end{frame}
\end{document}
