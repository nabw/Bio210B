\documentclass[14pt,aspectratio=169,xcolor=dvipsnames]{beamer}
\usetheme{SimplePlus}
\usepackage{booktabs}
\usepackage{minted}

\title[short title]{Clase 4: Módulos y Python Científico}
\subtitle{}
\author[NA Barnafi] {Nicolás Alejandro Barnafi Wittwer}
\institute[UC|CMM] 
{
    Pontificia Universidad Católica de Chile \\
    Centro de Modelamiento Matemático
}

\titlegraphic{
    \vspace{-1.8cm}
    \begin{flushright}
      \includegraphics[height=2.5cm]{../images/logos/puc.png} 
    \end{flushright}
}

\date{}

\begin{document}
%%%%%%%%%%%%%%%%%%%%%%%%%%%%%%%%%%%%%%%%%%%%%%%%%%%%%%%
\begin{frame}
    \maketitle
\end{frame}
%%%%%%%%%%%%%%%%%%%%%%%%%%%%%%%%%%%%%%%%%%%%%%%%%%%%%%%
%%%%%%%%%%%%%%%%%%%%%%%%%%%%%%%%%%%%%%%%%%%%%%%%%%%%%%%
\section{Módulos}
%%%%%%%%%%%%%%%%%%%%%%%%%%%%%%%%%%%%%%%%%%%%%%%%%%%%%%%
%%%%%%%%%%%%%%%%%%%%%%%%%%%%%%%%%%%%%%%%%%%%%%%%%%%%%%%
\begin{frame}[fragile]\frametitle{Motivación}
    Ya conocemos módulos: \code{string}, \code{sys}
    \begin{itemize}
        \item \code{import string}
        \item \code{import sys}
    \end{itemize}
    Así se cargan ciertas funcionalidades. 

    \pause Hoy aprenderemos a crear nuevos
\end{frame}
%%%%%%%%%%%%%%%%%%%%%%%%%%%%%%%%%%%%%%%%%%%%%%%%%%%%%%%
\begin{frame}[fragile]\frametitle{Pip}
    \textbf{P}ackage \textbf{I}nstaller for \textbf{P}ython
    
    \begin{itemize}
        \item Gestor de paquetes que viene instalado junto a Python
        \item Info online en \code{https://pypi.org/}
        \item Fácil de usar por Bash
            \begin{minted}{bash}
        $ pip install pandas
        $ python
        > import pandas  # recién instalada!
            \end{minted}
    \end{itemize}
    
    \pause Dónde están los programas...?
\end{frame}
%%%%%%%%%%%%%%%%%%%%%%%%%%%%%%%%%%%%%%%%%%%%%%%%%%%%%%%
\begin{frame}\frametitle{el PATH}
    Cuando escribo \code{ls} en Bash... de dónde sale? \\

    \vspace{0.5cm}
    La variable de entorno PATH tiene las carpetas donde buscar programas\footnote{Se puede modificar para tener programas en nuevas carpetas, i.e. \code{/opt}}

    \pause \idea{Consola...( echo \$\{PATH\} )}
\end{frame}
%%%%%%%%%%%%%%%%%%%%%%%%%%%%%%%%%%%%%%%%%%%%%%%%%%%%%%%
\begin{frame}[fragile]\frametitle{Crear módulo}
    \code{operaciones.py:}
    \begin{minted}{python}
  def suma(x,y): return x+y

  def mult(x,y): return x*y
    \end{minted}

  En Python:
  \begin{minted}{python}
  > import operaciones # Debe estar en el path
  > # from operaciones import suma,prod
  > operaciones.suma(2,3) # 5
  > operaciones.mult(2,3) # 6
  \end{minted}
\end{frame}
%%%%%%%%%%%%%%%%%%%%%%%%%%%%%%%%%%%%%%%%%%%%%%%%%%%%%%%
\begin{frame}[fragile]\frametitle{el PATH en Python}
    \begin{minted}{bash}
  ~/clases/Bio210B$ python -m site
   sys.path = [
       '/home/nico/clases/Bio210B',
       '/usr/lib/python3/dist-packages' # son más...
    \end{minted}

    \pause Se puede extender con PYTHONPATH
    \begin{minted}{bash}
  $ export PYTHONPATH=/home/nico/otraCarpeta
  $ python -m site
   sys.path = [
       '/home/nico/clases/Bio210B',
       '/home/nico/otraCarpeta',
       '/usr/lib/python3/dist-packages'
    \end{minted}

\end{frame}
%%%%%%%%%%%%%%%%%%%%%%%%%%%%%%%%%%%%%%%%%%%%%%%%%%%%%%%
\begin{frame}[fragile]\frametitle{Nota: variables en Bash}
    \begin{minted}{bash}
  $ export PYTHONPATH=[...]
    \end{minted}
    genera una variable de ambiente. Se puede ver con
    \begin{minted}{bash}
  $ echo ${PYTHONPATH}
    \end{minted}
  Todas las definidas se ven con programa \code{env} de Bas

\end{frame}
%%%%%%%%%%%%%%%%%%%%%%%%%%%%%%%%%%%%%%%%%%%%%%%%%%%%%%%
%%%%%%%%%%%%%%%%%%%%%%%%%%%%%%%%%%%%%%%%%%%%%%%%%%%%%%%
\section{Python científico}
%%%%%%%%%%%%%%%%%%%%%%%%%%%%%%%%%%%%%%%%%%%%%%%%%%%%%%%
%%%%%%%%%%%%%%%%%%%%%%%%%%%%%%%%%%%%%%%%%%%%%%%%%%%%%%%
\begin{frame}[fragile]\frametitle{Motivación}
    \begin{itemize}
        \item Python es interpretado $\to$ lento
        \item Pero Python usa librerías \emph{MUY rápidas}
        \item Eso + sintaxis lo vuelven ideal para la ciencia
    \end{itemize}
    \begin{center}
    \includegraphics[width=0.35\textwidth]{../images/baby-yoda-T.jpg} \hspace{1cm}\includegraphics[width=0.35\textwidth]{../images/baby-yoda-mesh.png}
    \end{center}
\end{frame}
%%%%%%%%%%%%%%%%%%%%%%%%%%%%%%%%%%%%%%%%%%%%%%%%%%%%%%%
\begin{frame}[fragile]\frametitle{\texttt{math}}
    \begin{minted}{python}
  > import math
  > math. # Tab
math.acos(   math.copysign(  math.expm1(      math.lgamma(     math.pi          math.tanh(
math.acosh(  math.cos(       math.fabs(       math.log(        math.pow(        math.tau
math.asin(   math.cosh(      math.factorial(  math.log10(      math.prod(       math.trunc(
math.asinh(  math.degrees(   math.floor(      math.log1p(      math.radians(    math.ulp(
math.atan(   math.dist(      math.fmod(       math.log2(       math.remainder(  
math.atan2(  math.e          math.frexp(      math.modf(       math.sin(        
math.atanh(  math.erf(       math.fsum(       math.nan         math.sinh(       
math.ceil(   math.erfc(      math.gamma(      math.nextafter(  math.sqrt(       
math.comb(   math.exp(       math.gcd(        math.perm(       math.tan(  
    \end{minted}
    
\end{frame}
%%%%%%%%%%%%%%%%%%%%%%%%%%%%%%%%%%%%%%%%%%%%%%%%%%%%%%%
\begin{frame}\frametitle{NumPy}
    \textbf{Num}erical \textbf{Py}thon
    
    \begin{itemize}
        \item Permite hacer operaciones a grandes datos
        \item Operaciones \emph{vectorizadas}
        \item Submódulos útiles: 
            \begin{itemize}
                \item numpy.fft (Fourier transform)
                \item numpy.linalg (Linear algebra)
                \item numpy.polynomial (Polinomios)
                \item numpy.random (Generador de números)
            \end{itemize}
    \end{itemize}
\end{frame}
%%%%%%%%%%%%%%%%%%%%%%%%%%%%%%%%%%%%%%%%%%%%%%%%%%%%%%%
\begin{frame}[fragile]\frametitle{Numpy arrays}
    \begin{minted}{python}
  > import numpy as np
  > # 50 números equiespaciados en [0,1]
  > num = np.linspace(0,1,50)
  > # Números en [0,1) con paso 0.1
  > num = np.arange(0,1,0.1)
    \end{minted}

\pause \idea{Veamos tiempos de ejecución en sumar}
\end{frame}
%%%%%%%%%%%%%%%%%%%%%%%%%%%%%%%%%%%%%%%%%%%%%%%%%%%%%%%
\begin{frame}[fragile]\frametitle{Operaciones vectorizadas}
    $$ \text{vector:} \qquad x\in \mathbb{R}^N \to \left.\begin{bmatrix} 1\\1.1\\\vdots \\ -0.1\end{bmatrix}\right\}\text{N números} $$

    \begin{minted}{python}
  > import numpy as np
  > num = np.linspace(0,1,50)
  > np.abs(num) # Val absoluto de todo
  > np.exp(num) # exponencial
  > np.sin(num) # seno, ... etc
  > num.dot(num) # normas, producto interno, etc
    \end{minted}

\end{frame}
%%%%%%%%%%%%%%%%%%%%%%%%%%%%%%%%%%%%%%%%%%%%%%%%%%%%%%%
\begin{frame}\frametitle{Matplotlib}
    \centering
    \only<1>{
    \includegraphics[width=0.6\textwidth]{../images/matplotlib/faces.png}
    }
    \only<2>{
    \includegraphics[width=0.5\textwidth]{../images/matplotlib/mandelbrot.png}
    }
    \only<3>{
    \includegraphics[width=0.5\textwidth]{../images/matplotlib/polygon.png}
    }
    \only<4>{
    \includegraphics[width=0.5\textwidth]{../images/matplotlib/time-series.png}
    }
    \only<5>{
    \includegraphics[width=0.5\textwidth]{../images/matplotlib/tricontourf.png}
    }
    \only<6>{
    \includegraphics[width=0.5\textwidth]{../images/matplotlib/3dstem.png}
    }
    \only<7>{
    \includegraphics[width=0.5\textwidth]{../images/matplotlib/3dvoxel.png}
    }
\end{frame}
%%%%%%%%%%%%%%%%%%%%%%%%%%%%%%%%%%%%%%%%%%%%%%%%%%%%%%%
\begin{frame}[fragile]\frametitle{Ejemplo simple}
    \begin{minted}{python}
  > import numpy as np
  > import matplotlib.pyplot as plt
  > xs = np.linspace(0, 2*np.pi, 1000) # [0,2 pi]
  > ys = np.sin(xs)
  > plt.plot(xs,ys) # Lista de puntos x e y
  > plt.show()
    \end{minted} 
   
    \pause Se puede agregar de todo... 

    Ver documentación online: \texttt{matplotlib.org}.
\end{frame}
%%%%%%%%%%%%%%%%%%%%%%%%%%%%%%%%%%%%%%%%%%%%%%%%%%%%%%%
\begin{frame}\frametitle{Recap}
    \begin{itemize}
        \item Dónde están los módulos por defecto
        \item Cómo agregar un módulo
        \item Cómo hacer módulo detectable con PYTHONPATH
        \item \texttt{math}
        \item \texttt{Numpy}
        \item \texttt{Matplotlib}
    \end{itemize}
\end{frame}
%%%%%%%%%%%%%%%%%%%%%%%%%%%%%%%%%%%%%%%%%%%%%%%%%%%%%%%
\begin{frame}
    \maketitle
\end{frame}
%%%%%%%%%%%%%%%%%%%%%%%%%%%%%%%%%%%%%%%%%%%%%%%%%%%%%%%
\begin{frame}[fragile]\frametitle{Mini ejercicios}
    \code{operaciones.py}
    \begin{minted}{python}
  def suma(x,y): return x+y

  def mult(x,y): return x*y
    \end{minted}

    \begin{itemize}
        \item Crear el módulo \code{operaciones} e importarlo desde un script.
        \item Guardar el módulo en otro lado e importarlo usando \code{PYTHONPATH}
    \end{itemize}

\end{frame}
%%%%%%%%%%%%%%%%%%%%%%%%%%%%%%%%%%%%%%%%%%%%%%%%%%%%%%%
\end{document}
