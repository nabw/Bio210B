\documentclass[14pt,aspectratio=169,xcolor=dvipsnames]{beamer}
\usetheme{SimplePlus}
\usepackage{booktabs}
\usepackage{minted}

\title[short title]{Clase 10: Scripts interactivo}
\subtitle{}
\author[NA Barnafi] {Nicolás Alejandro Barnafi Wittwer}
\institute[UC|CMM] 
{
    Pontificia Universidad Católica de Chile \\
    Centro de Modelamiento Matemático
}

\titlegraphic{
    \vspace{-1.8cm}
    \begin{flushright}
      \includegraphics[height=2.5cm]{../images/logos/puc.png} 
    \end{flushright}
}
\newcommand{\code}[1]{{\color{purple}\tt#1}}

\date{04/09/2024}
%\setbeamercovered{transparent}

\begin{document}
%%%%%%%%%%%%%%%%%%%%%%%%%%%%%%%%%%%%%%%%%%%%%%%%%%%%%%%
\begin{frame}
    \maketitle
\end{frame}
%%%%%%%%%%%%%%%%%%%%%%%%%%%%%%%%%%%%%%%%%%%%%%%%%%%%%%%
\begin{frame}[fragile]\frametitle{Motivación}
    \begin{minted}{python}
    a = 2
    textfile = "../carpeta/archivo.txt"
    
    # Usar 'a' y 'textfile'
    \end{minted}
    \begin{itemize}
        \item Poco portable
        \item Poco modular
        \item Ej: correr código para 1000 valores de 'a' en 15 archivos
    \end{itemize}

\pause \idea{Buscamos solución fuera del código}
\end{frame}
%%%%%%%%%%%%%%%%%%%%%%%%%%%%%%%%%%%%%%%%%%%%%%%%%%%%%%%
\begin{frame}[fragile]\frametitle{Input}
    Solución \#1: Frenar ejecución hasta que usuario entregue valor
    
    \begin{minted}{python}
    a = input("Ingrese valor de 'a': ")
    print(a, type(a))
    \end{minted}

    \vspace{1cm}
    El valor ingresado es un \code{str}
\end{frame}
%%%%%%%%%%%%%%%%%%%%%%%%%%%%%%%%%%%%%%%%%%%%%%%%%%%%%%%
\begin{frame}[fragile]\frametitle{Argv}
    Solución \#2: Usar parámetros de programa
    \begin{minted}{python}
    # programa.py
    from sys import argv
    name = argv[0]
    arg1 = argv[1]
    arg2 = argv[2]
    print("argv:",name, arg1, arg2)
    \end{minted}

    Luego en bash:
    \begin{minted}{bash}
    $ python programa.py 'hola' 42
    argv: programa.py hola 42
    \end{minted}
\end{frame}
%%%%%%%%%%%%%%%%%%%%%%%%%%%%%%%%%%%%%%%%%%%%%%%%%%%%%%%
\begin{frame}

\idea{Consola...}

\end{frame}
%%%%%%%%%%%%%%%%%%%%%%%%%%%%%%%%%%%%%%%%%%%%%%%%%%%%%%%
\begin{frame}\frametitle{Recap}
    \begin{itemize}
        \item Lectura a través de \emph{stream}
        \item Lectura de archivos con \code{open}
        \item Contexto con \code{with}
        \item Modos de lectura: \code{r},\code{w},\code{a}
    \end{itemize}
\end{frame}
%%%%%%%%%%%%%%%%%%%%%%%%%%%%%%%%%%%%%%%%%%%%%%%%%%%%%%%
\begin{frame}
    \maketitle
\end{frame}
\end{document}
