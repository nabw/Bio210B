\documentclass[14pt,aspectratio=169,xcolor=dvipsnames]{beamer}
\usetheme{SimplePlus}
\usepackage{booktabs}
\usepackage{minted}

\title[short title]{Clase 17: Documentación y docstrings}
\subtitle{}
\author[NA Barnafi] {Nicolás Alejandro Barnafi Wittwer}
\institute[UC|CMM] 
{
    Pontificia Universidad Católica de Chile \\
    Centro de Modelamiento Matemático
}

\titlegraphic{
    \vspace{-1.8cm}
    \begin{flushright}
      \includegraphics[height=2.5cm]{../images/logos/puc.png} 
    \end{flushright}
}

\date{16/10/2024}
%\setbeamercovered{transparent}

\begin{document}
%%%%%%%%%%%%%%%%%%%%%%%%%%%%%%%%%%%%%%%%%%%%%%%%%%%%%%%
\begin{frame}
    \maketitle
\end{frame}
%%%%%%%%%%%%%%%%%%%%%%%%%%%%%%%%%%%%%%%%%%%%%%%%%%%%%%%
\begin{frame}\frametitle{Motivación}
    \begin{itemize}
        \item El código \emph{envejece}
        \item Uno \textbf{no} recuerda por qué hizo cosas
        \item Los futuros usuarios de nuestro código \huge{menos}
    \end{itemize}
\end{frame}
%%%%%%%%%%%%%%%%%%%%%%%%%%%%%%%%%%%%%%%%%%%%%%%%%%%%%%%
\begin{frame}[fragile]\frametitle{Docs de una función}
    \begin{minted}{python}
    def suma(x,y):
        """
        Esta función entrega la suma de sus argumentos. 
            
        args: 
            x: Un elemento sumable
            y: Un segundo elemento sumable
        """
        z = x+y
        return z
    \end{minted}
\end{frame}
%%%%%%%%%%%%%%%%%%%%%%%%%%%%%%%%%%%%%%%%%%%%%%%%%%%%%%%
\begin{frame}[fragile]\frametitle{Docs de una función}
    \begin{minted}{python}
    >>> help(suma)
    \end{minted}

    \vspace{1cm}
    Se guarda en la variable mágica \code{suma.\_\_doc\_\_}

    \idea{Consola}
\end{frame}

%%%%%%%%%%%%%%%%%%%%%%%%%%%%%%%%%%%%%%%%%%%%%%%%%%%%%%%
\begin{frame}[fragile]\frametitle{Docs de una clase}
    \begin{footnotesize}
    \begin{minted}{python}
    class Vec2D:
        """
        Clase que representa un vector en R2.
        """
        def __init__(self,x,y):
            """
            Constructor, toma como parámetros coordenadas x e y.
            """
            self.x = x
            self.y = y
        def norma(self):
            """
            Entrega la norma euclideana del vector. Requiere 'sqrt'
            """
            return sqrt(self.x**2 + self.y**2)
    \end{minted}

    \vspace{-1cm}
    \hfill    \idea{Consola}
    \end{footnotesize}
\end{frame}
%%%%%%%%%%%%%%%%%%%%%%%%%%%%%%%%%%%%%%%%%%%%%%%%%%%%%%%
\begin{frame}[fragile]\frametitle{Documentación avanzada}
    Existe mucho software (Sphinx, Doxygen, etc). Veremos \code{Mkdocs}.

    \begin{minted}{bash}
    # pip install mkdocs
    # mkdocs new proyecto
    # cd proyecto
    # mkdocs serve
    # mkdocs build
    \end{minted} 

    \vspace{1cm}
    Se usa formato Markdown (popular por ser simple)

\end{frame}
%%%%%%%%%%%%%%%%%%%%%%%%%%%%%%%%%%%%%%%%%%%%%%%%%%%%%%%
\darkSlide{Veamos su documentación: \\https://www.mkdocs.org/getting-started/}
%%%%%%%%%%%%%%%%%%%%%%%%%%%%%%%%%%%%%%%%%%%%%%%%%%%%%%%
\begin{frame}[fragile]\frametitle{Otra opción}
    \begin{minted}{bash}
    # pip install pdoc3
    # cd proyecto
    # pdoc3 --html .
    \end{minted}

    \idea{Consola}
\end{frame}
%%%%%%%%%%%%%%%%%%%%%%%%%%%%%%%%%%%%%%%%%%%%%%%%%%%%%%%
\begin{frame}\frametitle{Recap}
    \begin{itemize}
        \item Documentación es fundamental
        \item Existen muchas herramientas para hacerlo en modo semi-automático
        \item \code{mkdocs} bonito pero trabajoso
        \item \code{pdoc3} feo pero fácil
    \end{itemize}
\end{frame}
%%%%%%%%%%%%%%%%%%%%%%%%%%%%%%%%%%%%%%%%%%%%%%%%%%%%%%%
\begin{frame}
    \maketitle
\end{frame}
%%%%%%%%%%%%%%%%%%%%%%%%%%%%%%%%%%%%%%%%%%%%%%%%%%%%%%%
\begin{frame}[noframenumbering]\frametitle{Mini ejercicios}
    \begin{itemize}
        \item Crear una carpeta de nombre 'proyecto'
        \item Dentro esa carpeta crear un archivo con dos funciones
        \item Instalar pdoc3 y correr \code{pdoc3 --html .} desde la carpeta
        \item Ver documentación en archivo html recién creado
    \end{itemize}
\end{frame}
%%%%%%%%%%%%%%%%%%%%%%%%%%%%%%%%%%%%%%%%%%%%%%%%%%%%%%%
\end{document}
