\documentclass[14pt,aspectratio=169,xcolor=dvipsnames]{beamer}
\usetheme{SimplePlus}
\usepackage{booktabs}
\usepackage{minted}

\title[short title]{Clase 22: Bash como orquestador II}
\subtitle{}
\author[NA Barnafi] {Nicolás Alejandro Barnafi Wittwer}
\institute[UC|CMM] 
{
    Pontificia Universidad Católica de Chile \\
    Centro de Modelamiento Matemático
}

\titlegraphic{
    \vspace{-1.8cm}
    \begin{flushright}
      \includegraphics[height=2.5cm]{../images/logos/puc.png} 
    \end{flushright}
}

\date{16/10/2024}
%\setbeamercovered{transparent}

\begin{document}
%%%%%%%%%%%%%%%%%%%%%%%%%%%%%%%%%%%%%%%%%%%%%%%%%%%%%%%
\begin{frame}
    \maketitle
\end{frame}
%%%%%%%%%%%%%%%%%%%%%%%%%%%%%%%%%%%%%%%%%%%%%%%%%%%%%%%
\begin{frame}\frametitle{Motivación}
    \begin{itemize}
        \item Desde \code{bash} podemos ejecutar cualquier programa
        \item En particular \emph{programas de python}
        \item Bash maneja entradas y salidas de texto
        \item \code{echo} es nuestro \code{print} de \code{bash}
    \end{itemize}
\end{frame}
%%%%%%%%%%%%%%%%%%%%%%%%%%%%%%%%%%%%%%%%%%%%%%%%%%%%%%%
\begin{frame}[fragile]{Pipes y redirecciones}
    \begin{small}
    \begin{minted}{bash}
  $ echo "hola"  # Imprime 'hola'
  $ # Toma el texto generado por 'echo' y lo pone en 'out.txt'
  $ echo "hola" > out.txt 
  $ # Toma el texto y lo pone al final de 'out.txt'
  $ echo "hola" > out.txt 
  $ cat 'out.txt' | grep "ho"
    \end{minted} 
    \end{small}
La última línea: toma la salida de \code{cat out.txt} y se lo entrega como input a la función \code{grep}.

\idea{Consola}
\end{frame}
%%%%%%%%%%%%%%%%%%%%%%%%%%%%%%%%%%%%%%%%%%%%%%%%%%%%%%%
\begin{frame}[fragile]\frametitle{Orquestación: Ejercicio}
    Creemos: 
    \begin{enumerate}
        \item Un programa en Python que calcula el promedio de números dados
        \item Llamemos este programa desde un script en Bash
        \item Hagamos que el script en Bash abra un archivo de números y los de a Python
        \item Mostrar resultado final
    \end{enumerate}
\end{frame}
%%%%%%%%%%%%%%%%%%%%%%%%%%%%%%%%%%%%%%%%%%%%%%%%%%%%%%%
\begin{frame}\frametitle{Recap}
    \begin{itemize}
        \item En Python/R es fácil crear funciones específicas
        \item Estas funciones se pueden orquestar desde Bash
        \item Bash tiene muchas funcionalidades para manipular texto
        \item ... esto es solo la punta del iceberg
    \end{itemize}
\end{frame}
%%%%%%%%%%%%%%%%%%%%%%%%%%%%%%%%%%%%%%%%%%%%%%%%%%%%%%%
\begin{frame}
    \maketitle
\end{frame}
%%%%%%%%%%%%%%%%%%%%%%%%%%%%%%%%%%%%%%%%%%%%%%%%%%%%%%%
\begin{frame}[noframenumbering]\frametitle{Mini ejercicios}
    Creemos: 
    \begin{enumerate}
        \item Un programa en Python que calcula el promedio de números dados
        \item Llamemos este programa desde un script en Bash
        \item Hagamos que el script en Bash abra un archivo de números y los de a Python
        \item Mostrar resultado final
    \end{enumerate}
\end{frame}
%%%%%%%%%%%%%%%%%%%%%%%%%%%%%%%%%%%%%%%%%%%%%%%%%%%%%%%
\end{document}
