\documentclass{article}
\usepackage{fullpage}
\usepackage{mathpazo}
\usepackage{todonotes}
\usepackage{minted}

\newcommand{\note}[1]{\todo[inline,color=gray!20!white]{#1}}

\title{Tarea 2}
\author{Nicolás Barnafi, Humberto Reyes}
\date{28 / 10 / 2024}

\begin{document}
\maketitle

\note{\textbf{Instrucciones:} Esta tarea se puede hacer en parejas. Para su entrega, debe crear un repositorio en Github que deberá compartir con el profesor (usuario 'nabw'). La entrega es para el viernes 01 de noviembre hasta las 23:59, y desde esa hora se aplicará un descuento lineal que llega a 0.0 puntos después de 48 horas.}

\begin{itemize}
    \item[\textbf{3.0 puntos}] El objetivo de esta pregunta es desarrollar un código que lea datos con una cierta estructura y graficarlos. El resultado final será tener un script en Python llamado 'p1.py' que puede ser llamado desde Bash de la siguiente manera:
    
    \begin{minted}{bash}
    # python p1.py "archivo.csv" "out.png"
    \end{minted}

    donde "archivo.csv" es el nombre del archivo por leer y "out.png" es el nombre de la figura que se exportará. El archivo "archivo.csv" es un archivo en formato csv que puede tener dos o tres columnas, donde la primera representa un instante y las siguientes representan una posición. En el caso de dos columnas, representa una posición en el eje y, y en el caso de tres columnas se tiene que representan las coordenadas x e y respectivamente. La primera fila será el nombre de las variables. Un ejemplo es el siguiente: 

\begin{verbatim}
"tiempo","v1","v2"
0.0,0,2
1.0,4.0,-27
\end{verbatim}


    \begin{itemize}
        \item[1 punto] Cree un módulo llamado 'read.py' que contenga una función 'read' tal que recibe un string con el nombre del archivo y que entregue dos listas: una con los números de la primera columna y otra con los números de las columnas restantes (o sea, una lista de números en el caso de dos columnas y dos listas en el caso de dos columnas). Cree además un archivo 'p1-1.py' que se llama en Bash usando

    \begin{minted}{bash}
    # python p1-1.py "archivo.csv"
    \end{minted}

        que use el módulo 'read' y que, a través de la función implementada, lea el archivo entregado y que imprima los primeros 5 elementos de cada lista. Si los elementos son menos de 5, entonces debe imprimir todos los elementos.

        \item Cree un módulo 'plot.py' que implemente dos funciones: (1) una función 'plot1D' que recibe dos listas y dos strings. Los strings serán los nombres de los ejes, y las listas serán los datos para los ejes x e y respectivamente. (2) una función 'plot2D' que recibe tres listas y dos strings. Los dos strings serán los nombres de los ejes, y las listas serán las coordenadas x, las coordenadas y, y el valor de un campo en cada punto (por ejemplo, el tiempo en este caso). La función 1D hará un gráfico con la función 'plot' de matplotlib', y la función 2D lo hará con la función 'scatter'. Ojo que los valores en el caso 2D corresponden al parámetro 'c' de scatter. Para probarlo, cree un archivo 'p1-2.py' llamado en Bash con

    \begin{minted}{bash}
    # python p1-2.py "out1d.png" "out2d.png"
    \end{minted}
 
        donde "out1d.png" y "out2d.png" son dos parámtetros de su script con los nombres de los archivos para exportar. Usando las funciones plot1D y plot2D, que grafique los siguientes datos: 

    \begin{minted}{python}
    # Datos para plot1D
    t1 = [0,1, 2,3, 4,5, 6,7, 8]
    x1 = [0,1,-1,1,-1,1,-1,1,-1]

    # Datos para plot2D
    t2 = [0. , 0.33, 0.66, 0.99, 1.32, 
          1.65, 1.98, 2.31, 2.64, 2.98, 
          3.31, 3.64 , 3.97, 4.30, 4.63, 
          4.96, 5.29, 5.62, 5.95, 6.28]
    x2 = [ 1. , 0.95, 0.79, 0.55, 0.25, 
          -0.08, -0.4 , -0.68, -0.88, -0.99, 
          -0.99, -0.88, -0.68, -0.4 , -0.08, 
           0.25, 0.55, 0.79, 0.95,  1.  ]
    y2 = [ 0.00e+00,  3.25e-01,  6.14e-01,  8.37e-01,  9.69e-01, 
           9.97e-01,  9.16e-01,  7.36e-01,  4.76e-01,  1.65e-01, 
          -1.65e-01, -4.76e-01, -7.36e-01, -9.16e-01, -9.97e-01, 
          -9.69e-01, -8.37e-01, -6.14e-01, -3.25e-01, -2.45e-16]
    \end{minted}

        \item A partir de los módulos 'plot' y 'read', cree el archivo 'p1.py' descrito al inicio de este ejercicio. La evaluación consistirá en probar la función con datos de distintos de tamaños para cada caso.

    \end{itemize} % Preguntas de P1
    
    \item[\textbf{3.0 puntos}] En este ejercicio, implementará el juego de la vida de Conway. Para ello, organizará su código de la siguiente manera:
        \begin{itemize}
            \item Cree una clase 'Celda' con las siguientes funcionalidades: (1) Posee un parámetro 'estado' que dice si la celda está viva o muerta, (2) su constructor recibe como parámetro el estado inicial, y por defecto el estado es muerto. (3) una función 'interactuar' que recibe una lista de celdas y sigue las reglas del juego de Conway: 
                \begin{itemize}
                    \item Si la celda está viva y la lista posee menos de dos elementos vivos, entonces la celda muere.
                    \item Si la celda está viva y la lista posee dos o tres vivas, entonces sigue viva.
                    \item Si la celda está viva y la lista posee más de tres celdas vivas, entonces muere.
                    \item Si la celda está muerta y recibe exactamente tres celdas vivas, entonces vive.
                \end{itemize}

            \item Cree una clase Grilla con las siguientes funcionalidades: 
                \begin{itemize}
                    \item Un constructor que recibe un número entero que representa el tamaño de grilla ($N\times N$) y una lista de tuplas de números enteros (todos entre 0 y $N-1$) que representan las posiciones de las celdas que están inicialmente vivas. El constructor deberá crear dos listas de listas de celdas que representan las celdas en cada posición de la grilla llamadas 'celdas' y 'celdas\_siguiente'. Por ejemplo, si la grilla es de $30\times 30$, entonces la celda en la posición (4,7) se obtiene con 

                    \begin{minted}{python}
       grilla.celdas[3][6] 
                    \end{minted}
                    El objetivo de la lista 'siguiente' es poder actualizar los estados de las celdas sin sobreescribir el estado de las celdas actuales. Finalmente, la grilla deberá tener dos parámetros: uno con un string que indica el nombre de archivo para exportar y otro con un contador. Esto será útil para exportar las soluciones actuales.
                    \item Una función 'actualizar\_celdas', que no recibe ningún parámetro y que lo que hace es que copia el estado de todas las celdas en 'celdas\_siguiente' en el estado de las celdas en 'celdas'. Es importante que copie el estado de las celdas y no las celdas mismas haciendo 'grilla.celdas[i][j] = grilla.celdas\_siguiente[i][j]'. Si hace eso, hará que ambas listas hagan referencia a la misma memoria, y el funcionamiento del código será errático e incorrecto.
                    \item Una función 'visualizar' que tome una lista de listas de celdas y que (1) la transforme en un numpy.array de dos ejes de 0's y 1's y luego (2) que exporte un gráfico de dicho numpy.array. Para generar el gráfico, será conveniente revisar la función 'imshow' de matplotlib.
                    \item Una función 'avanzar', que haga las siguientes acciones: (1) llamar a 'actualizar\_celdas' para descartar el estado actual de 'celdas' y considerar como estado actual el de las celdas siguientes, (2) para todas las celdas de la grilla, deberá actualizar en estado de las celdas en 'celdas\_siguiente' entregándole como input, para cada celda todas sus celdas vecinas. Las celdas vecinas se definen como todas las celdas que están directamente al lado de la celda en sentido horizontal, vertical, y diagonal. Esto implica que todas las celdas (excepto las del borde) tienen 9 vecinas.
                \end{itemize} % Funcionalidades de Grilla
            \item Alternativamente, puede implementar el juego de la vida sin clases, con lo cual aspira a un máximo de 1.5 punto si es que el código funciona.
        \end{itemize} % Orden de Conway
    Para evaluar este ejercicio, deberá crear una rutina 'conway.py' que se llama con 

    \begin{minted}{bash}
    # python conway.py "datos.ini"
    \end{minted}

    El archivo 'datos.ini' tendrá el siguiente formato:

\begin{verbatim}
30
1,2
3,5
20,1
\end{verbatim}

    donde la primera fila dice el tamaño de la grilla por simular, y el resto de las filas dice las posiciones de las celdas que están inicialmente vivas. 

\end{itemize} % Ejercicios

\note{Notar que no está prohibido usar Chat GPT para desarollar esta tarea. Sin embargo, para asegurarnos de que el código entregado fue desarrollado a mano, elegiremos un grupo al azar de personas que tendrán que explicar su código línea por línea. Falta de comprensión en el código entregado será tratado como si fuese copia.}

\end{document}
