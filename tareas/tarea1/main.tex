\documentclass{article}
\usepackage{fullpage}
\usepackage{mathpazo}
\usepackage{todonotes}
\usepackage{minted}

\newcommand{\note}[1]{\todo[inline,color=gray!20!white]{#1}}

\title{Tarea 1}
\author{Nicolás Barnafi, Humberto Reyes}
\date{30 / 08 / 2024}

\begin{document}
\maketitle

\note{\textbf{Instrucciones:} Para la entrega de esta tarea, debe enviar por Canvas en el formulario habilitado un archivo comprimido \texttt{zip} con un archivo \texttt{.py} para cada ejercicio. La entrega es para el lunes 09 de septiembre a las 18:00, y desde esa hora se aplicará un descuento lineal tal que todas las entregas desde el martes 10 a las 18:00 tengan nota 1.0. Por esta vez, no será importante que la tearea entregada tenga comentarios que ayuden a guiar la corrección.}

\begin{itemize}
    \item[\textbf{2.0 puntos} (1)] El objetivo de este ejercicio es generar un pequeño programa que pueda convertir unidades. El objetivo será que en las primeras dos líneas del código el usuario escriba dos variables predeterminadas, una variable con una cantidad (un número) y otra con un string que indique las unidades. Por ejemplo, el inicio de su programa podría ser:
        \begin{minted}{python}
            a = 127.8
            b = "Pa"
        \end{minted}
    En este caso, la variable 'a' indica una cantidad, y la variable 'b' indica que esa cantidad está en \textbf{Pa}scales, y por lo tanto es una medida de presión. Su código debe cumplir con los siguientes requisitos:
        \begin{itemize}
            \item La variable de unidades puede ser una de las siguientes: 'Pa' (Pascales), 'mmHg' (Milímetros de mercurio), 'g' (gramos), 'kg' (kilogramos), 'mg' (miligramos). Cualquier otro valor debe dar un mensaje de error y detener la ejecución del código.
            \item La variable de cantidad debe ser un número positivo. Si este número es negativo, el programa debe dar un mensaje de error y detener su ejecución.
            \item Debe implementar las siguientes funcionalidades, implementadas en funciones separadas: 
                \begin{itemize}
                    \item Convertir de Pascales a mmHg y vice-versa.
                    \item Convertir de gramos a miligramos.
                    \item Convertir de gramos a kilogramos.
                    \item Convertir de miligramos a kilogramos.
                \end{itemize}
                Llamar a cualquiera de estas funciones con unidades que no corresponda (ej: de Pascales a gramos) debe dar un error. 
        \end{itemize}
    La evaluación de este ejercicio consistirá en tomar el script que entreguen y agregar al final (i) funciones para convertir entre las distintas unidades de masa, (ii) funciones para convertir entre presiones, y (iii) funciones con entradas equivocadas para probar los distintos mensajes de error. 
    \item[\textbf{2.0 puntos} (2)] Implemente un script en Python tal que, dado un \texttt{string} que consiste en una cadena de bases nitrogenadas:
        \begin{minted}{python}
            genoma = "ACTGCTGACTAGACGATAGC"
        \end{minted}
    debe poder realizar las siguientes acciones a través de funciones específicas para cada una: 
        \begin{itemize}
            \item Contar el número de adeninas
            \item Contar el número de citocinas
            \item Contar el número de guaninas
            \item Contar el número de timinas
            \item Entregar la hebra de ADN conjugada
        \end{itemize}
    La evaluación de este ejercicio consistirá en modificar el genoma al inicio del archivo y evaluar que las salidas sean correctas. 
    
    \item[\textbf{2.0 puntos} (3)] Este ejercicio consiste en estudiar el concepto de \emph{list comprehensions}, que es una manera cómoda y eficiente de generar listas. La idea es tomar un iterador, y poner inmediatamente cuales serían los elementos resultantes de la lista. Por ejemplo, esta línea de código genera todos los números pares entre 1 y 10: 
        \begin{minted}{python}
            pares = [ 2*i for i in range(1,6) ] 
        \end{minted}
    Note que los números considerados en el iterador \texttt{range} son \texttt{[1,2,3,4,5]}, y por lo tanto el resultado de este código sería que la variable \texttt{pares} valga \texttt{[2,4,6,8,10]}. Usando \emph{list comprehensions}, genere las siguientes listas: 
        \begin{itemize}
            \item Todos los números impares entre 1 y 7000
            \item Una lista de tuplas de largo 2, donde el primer elemento serán todos los números entre 1 y 100, y el siguiente es del primer número al cuadrado. Para mayor claridad, el primer elemento de la lista debiese $(1,1^2)$, que es $(1,1)$. 
            \item Una lista de tuplas de largo 2, donde el primer elemento serán todos los números entre 1 y 100, y el siguiente es un string que dice "Este número es X", donde X es el valor el primer elemento. Para mayor claridad, el primer elemento de la lista debiese ser \texttt{(1, "Este número es 1")}.
        \end{itemize}

\end{itemize}

\note{Notar que no está prohibido usar Chat GPT para desarollar esta tarea. Sin embargo, para asegurarnos de que el código entregado fue desarrollado a mano, elegiremos un grupo al azar de 6 personas que tendrán que explicar su código línea por línea. Falta de comprensión en el código entregado será tratado como si fuese copia.}

\end{document}
